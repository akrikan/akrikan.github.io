% Options for packages loaded elsewhere
\PassOptionsToPackage{unicode}{hyperref}
\PassOptionsToPackage{hyphens}{url}
\PassOptionsToPackage{dvipsnames,svgnames,x11names}{xcolor}
%
\documentclass[
  letterpaper,
  DIV=11,
  numbers=noendperiod]{scrartcl}

\usepackage{amsmath,amssymb}
\usepackage{lmodern}
\usepackage{iftex}
\ifPDFTeX
  \usepackage[T1]{fontenc}
  \usepackage[utf8]{inputenc}
  \usepackage{textcomp} % provide euro and other symbols
\else % if luatex or xetex
  \usepackage{unicode-math}
  \defaultfontfeatures{Scale=MatchLowercase}
  \defaultfontfeatures[\rmfamily]{Ligatures=TeX,Scale=1}
\fi
% Use upquote if available, for straight quotes in verbatim environments
\IfFileExists{upquote.sty}{\usepackage{upquote}}{}
\IfFileExists{microtype.sty}{% use microtype if available
  \usepackage[]{microtype}
  \UseMicrotypeSet[protrusion]{basicmath} % disable protrusion for tt fonts
}{}
\makeatletter
\@ifundefined{KOMAClassName}{% if non-KOMA class
  \IfFileExists{parskip.sty}{%
    \usepackage{parskip}
  }{% else
    \setlength{\parindent}{0pt}
    \setlength{\parskip}{6pt plus 2pt minus 1pt}}
}{% if KOMA class
  \KOMAoptions{parskip=half}}
\makeatother
\usepackage{xcolor}
\usepackage[top=1in,left=1in,right=1in,bottom=1in]{geometry}
\setlength{\emergencystretch}{3em} % prevent overfull lines
\setcounter{secnumdepth}{5}
% Make \paragraph and \subparagraph free-standing
\ifx\paragraph\undefined\else
  \let\oldparagraph\paragraph
  \renewcommand{\paragraph}[1]{\oldparagraph{#1}\mbox{}}
\fi
\ifx\subparagraph\undefined\else
  \let\oldsubparagraph\subparagraph
  \renewcommand{\subparagraph}[1]{\oldsubparagraph{#1}\mbox{}}
\fi

\usepackage{color}
\usepackage{fancyvrb}
\newcommand{\VerbBar}{|}
\newcommand{\VERB}{\Verb[commandchars=\\\{\}]}
\DefineVerbatimEnvironment{Highlighting}{Verbatim}{commandchars=\\\{\}}
% Add ',fontsize=\small' for more characters per line
\usepackage{framed}
\definecolor{shadecolor}{RGB}{241,243,245}
\newenvironment{Shaded}{\begin{snugshade}}{\end{snugshade}}
\newcommand{\AlertTok}[1]{\textcolor[rgb]{0.68,0.00,0.00}{#1}}
\newcommand{\AnnotationTok}[1]{\textcolor[rgb]{0.37,0.37,0.37}{#1}}
\newcommand{\AttributeTok}[1]{\textcolor[rgb]{0.40,0.45,0.13}{#1}}
\newcommand{\BaseNTok}[1]{\textcolor[rgb]{0.68,0.00,0.00}{#1}}
\newcommand{\BuiltInTok}[1]{\textcolor[rgb]{0.00,0.23,0.31}{#1}}
\newcommand{\CharTok}[1]{\textcolor[rgb]{0.13,0.47,0.30}{#1}}
\newcommand{\CommentTok}[1]{\textcolor[rgb]{0.37,0.37,0.37}{#1}}
\newcommand{\CommentVarTok}[1]{\textcolor[rgb]{0.37,0.37,0.37}{\textit{#1}}}
\newcommand{\ConstantTok}[1]{\textcolor[rgb]{0.56,0.35,0.01}{#1}}
\newcommand{\ControlFlowTok}[1]{\textcolor[rgb]{0.00,0.23,0.31}{#1}}
\newcommand{\DataTypeTok}[1]{\textcolor[rgb]{0.68,0.00,0.00}{#1}}
\newcommand{\DecValTok}[1]{\textcolor[rgb]{0.68,0.00,0.00}{#1}}
\newcommand{\DocumentationTok}[1]{\textcolor[rgb]{0.37,0.37,0.37}{\textit{#1}}}
\newcommand{\ErrorTok}[1]{\textcolor[rgb]{0.68,0.00,0.00}{#1}}
\newcommand{\ExtensionTok}[1]{\textcolor[rgb]{0.00,0.23,0.31}{#1}}
\newcommand{\FloatTok}[1]{\textcolor[rgb]{0.68,0.00,0.00}{#1}}
\newcommand{\FunctionTok}[1]{\textcolor[rgb]{0.28,0.35,0.67}{#1}}
\newcommand{\ImportTok}[1]{\textcolor[rgb]{0.00,0.46,0.62}{#1}}
\newcommand{\InformationTok}[1]{\textcolor[rgb]{0.37,0.37,0.37}{#1}}
\newcommand{\KeywordTok}[1]{\textcolor[rgb]{0.00,0.23,0.31}{#1}}
\newcommand{\NormalTok}[1]{\textcolor[rgb]{0.00,0.23,0.31}{#1}}
\newcommand{\OperatorTok}[1]{\textcolor[rgb]{0.37,0.37,0.37}{#1}}
\newcommand{\OtherTok}[1]{\textcolor[rgb]{0.00,0.23,0.31}{#1}}
\newcommand{\PreprocessorTok}[1]{\textcolor[rgb]{0.68,0.00,0.00}{#1}}
\newcommand{\RegionMarkerTok}[1]{\textcolor[rgb]{0.00,0.23,0.31}{#1}}
\newcommand{\SpecialCharTok}[1]{\textcolor[rgb]{0.37,0.37,0.37}{#1}}
\newcommand{\SpecialStringTok}[1]{\textcolor[rgb]{0.13,0.47,0.30}{#1}}
\newcommand{\StringTok}[1]{\textcolor[rgb]{0.13,0.47,0.30}{#1}}
\newcommand{\VariableTok}[1]{\textcolor[rgb]{0.07,0.07,0.07}{#1}}
\newcommand{\VerbatimStringTok}[1]{\textcolor[rgb]{0.13,0.47,0.30}{#1}}
\newcommand{\WarningTok}[1]{\textcolor[rgb]{0.37,0.37,0.37}{\textit{#1}}}

\providecommand{\tightlist}{%
  \setlength{\itemsep}{0pt}\setlength{\parskip}{0pt}}\usepackage{longtable,booktabs,array}
\usepackage{calc} % for calculating minipage widths
% Correct order of tables after \paragraph or \subparagraph
\usepackage{etoolbox}
\makeatletter
\patchcmd\longtable{\par}{\if@noskipsec\mbox{}\fi\par}{}{}
\makeatother
% Allow footnotes in longtable head/foot
\IfFileExists{footnotehyper.sty}{\usepackage{footnotehyper}}{\usepackage{footnote}}
\makesavenoteenv{longtable}
\usepackage{graphicx}
\makeatletter
\def\maxwidth{\ifdim\Gin@nat@width>\linewidth\linewidth\else\Gin@nat@width\fi}
\def\maxheight{\ifdim\Gin@nat@height>\textheight\textheight\else\Gin@nat@height\fi}
\makeatother
% Scale images if necessary, so that they will not overflow the page
% margins by default, and it is still possible to overwrite the defaults
% using explicit options in \includegraphics[width, height, ...]{}
\setkeys{Gin}{width=\maxwidth,height=\maxheight,keepaspectratio}
% Set default figure placement to htbp
\makeatletter
\def\fps@figure{htbp}
\makeatother

\KOMAoption{captions}{tableheading}
\makeatletter
\@ifpackageloaded{tcolorbox}{}{\usepackage[many]{tcolorbox}}
\@ifpackageloaded{fontawesome5}{}{\usepackage{fontawesome5}}
\definecolor{quarto-callout-color}{HTML}{909090}
\definecolor{quarto-callout-note-color}{HTML}{0758E5}
\definecolor{quarto-callout-important-color}{HTML}{CC1914}
\definecolor{quarto-callout-warning-color}{HTML}{EB9113}
\definecolor{quarto-callout-tip-color}{HTML}{00A047}
\definecolor{quarto-callout-caution-color}{HTML}{FC5300}
\definecolor{quarto-callout-color-frame}{HTML}{acacac}
\definecolor{quarto-callout-note-color-frame}{HTML}{4582ec}
\definecolor{quarto-callout-important-color-frame}{HTML}{d9534f}
\definecolor{quarto-callout-warning-color-frame}{HTML}{f0ad4e}
\definecolor{quarto-callout-tip-color-frame}{HTML}{02b875}
\definecolor{quarto-callout-caution-color-frame}{HTML}{fd7e14}
\makeatother
\makeatletter
\makeatother
\makeatletter
\makeatother
\makeatletter
\@ifpackageloaded{caption}{}{\usepackage{caption}}
\AtBeginDocument{%
\ifdefined\contentsname
  \renewcommand*\contentsname{Table of contents}
\else
  \newcommand\contentsname{Table of contents}
\fi
\ifdefined\listfigurename
  \renewcommand*\listfigurename{List of Figures}
\else
  \newcommand\listfigurename{List of Figures}
\fi
\ifdefined\listtablename
  \renewcommand*\listtablename{List of Tables}
\else
  \newcommand\listtablename{List of Tables}
\fi
\ifdefined\figurename
  \renewcommand*\figurename{Figure}
\else
  \newcommand\figurename{Figure}
\fi
\ifdefined\tablename
  \renewcommand*\tablename{Table}
\else
  \newcommand\tablename{Table}
\fi
}
\@ifpackageloaded{float}{}{\usepackage{float}}
\floatstyle{ruled}
\@ifundefined{c@chapter}{\newfloat{codelisting}{h}{lop}}{\newfloat{codelisting}{h}{lop}[chapter]}
\floatname{codelisting}{Listing}
\newcommand*\listoflistings{\listof{codelisting}{List of Listings}}
\makeatother
\makeatletter
\@ifpackageloaded{caption}{}{\usepackage{caption}}
\@ifpackageloaded{subcaption}{}{\usepackage{subcaption}}
\makeatother
\makeatletter
\@ifpackageloaded{tcolorbox}{}{\usepackage[many]{tcolorbox}}
\makeatother
\makeatletter
\@ifundefined{shadecolor}{\definecolor{shadecolor}{rgb}{.97, .97, .97}}
\makeatother
\makeatletter
\makeatother
\ifLuaTeX
  \usepackage{selnolig}  % disable illegal ligatures
\fi
\IfFileExists{bookmark.sty}{\usepackage{bookmark}}{\usepackage{hyperref}}
\IfFileExists{xurl.sty}{\usepackage{xurl}}{} % add URL line breaks if available
\urlstyle{same} % disable monospaced font for URLs
\hypersetup{
  pdftitle={Complete Black-Scholes Proofs: From First Principles},
  colorlinks=true,
  linkcolor={blue},
  filecolor={Maroon},
  citecolor={Blue},
  urlcolor={Blue},
  pdfcreator={LaTeX via pandoc}}

\title{Complete Black-Scholes Proofs: From First Principles}
\author{}
\date{5/29/25}

\begin{document}
\maketitle
\ifdefined\Shaded\renewenvironment{Shaded}{\begin{tcolorbox}[borderline west={3pt}{0pt}{shadecolor}, sharp corners, interior hidden, enhanced, frame hidden, boxrule=0pt, breakable]}{\end{tcolorbox}}\fi

\renewcommand*\contentsname{Table of contents}
{
\hypersetup{linkcolor=}
\setcounter{tocdepth}{3}
\tableofcontents
}
\hypertarget{part-i-mathematical-foundations-and-definitions}{%
\section{Part I: Mathematical Foundations and
Definitions}\label{part-i-mathematical-foundations-and-definitions}}

This section establishes all the mathematical machinery needed for the
Black-Scholes proofs. We assume knowledge of measure theory and basic
stochastic calculus but will define all financial and probabilistic
concepts explicitly.

\hypertarget{financial-market-framework}{%
\subsection{Financial Market
Framework}\label{financial-market-framework}}

\begin{tcolorbox}[enhanced jigsaw, colback=white, rightrule=.15mm, coltitle=black, left=2mm, colframe=quarto-callout-note-color-frame, breakable, arc=.35mm, bottomtitle=1mm, toptitle=1mm, titlerule=0mm, leftrule=.75mm, colbacktitle=quarto-callout-note-color!10!white, title={Definition 1.1 (Financial Market)}, toprule=.15mm, opacitybacktitle=0.6, bottomrule=.15mm, opacityback=0]

A financial market consists of:

\begin{enumerate}
\def\labelenumi{\arabic{enumi}.}
\tightlist
\item
  A probability space \((\Omega, \mathcal{F}, \mathbb{P})\) where
  \(\Omega\) represents all possible market scenarios
\item
  A filtration \(\{\mathcal{F}_t\}_{t \geq 0}\) representing information
  available at time \(t\)
\item
  A finite collection of traded assets with price processes
  \(\{S^i_t\}_{i=0,1,\ldots,n}\)
\end{enumerate}

\end{tcolorbox}

\begin{tcolorbox}[enhanced jigsaw, colback=white, rightrule=.15mm, coltitle=black, left=2mm, colframe=quarto-callout-note-color-frame, breakable, arc=.35mm, bottomtitle=1mm, toptitle=1mm, titlerule=0mm, leftrule=.75mm, colbacktitle=quarto-callout-note-color!10!white, title={Definition 1.2 (Brownian Motion)}, toprule=.15mm, opacitybacktitle=0.6, bottomrule=.15mm, opacityback=0]

A stochastic process \(\{W_t\}_{t \geq 0}\) on
\((\Omega, \mathcal{F}, \mathbb{P})\) is a standard Brownian motion if:

\begin{enumerate}
\def\labelenumi{\arabic{enumi}.}
\tightlist
\item
  \(W_0 = 0\) almost surely
\item
  \(W\) has independent increments: for \(0 \leq s < t\), \(W_t - W_s\)
  is independent of \(\mathcal{F}_s\)
\item
  \(W_t - W_s \sim N(0, t-s)\) for all \(0 \leq s < t\)
\item
  \(W\) has continuous sample paths almost surely
\end{enumerate}

\end{tcolorbox}

\begin{tcolorbox}[enhanced jigsaw, colback=white, rightrule=.15mm, coltitle=black, left=2mm, colframe=quarto-callout-note-color-frame, breakable, arc=.35mm, bottomtitle=1mm, toptitle=1mm, titlerule=0mm, leftrule=.75mm, colbacktitle=quarto-callout-note-color!10!white, title={Definition 1.3 (Natural Filtration of Brownian Motion)}, toprule=.15mm, opacitybacktitle=0.6, bottomrule=.15mm, opacityback=0]

The natural filtration of Brownian motion is
\[\mathcal{F}_t^W = \sigma(W_s : 0 \leq s \leq t)\] the
\(\sigma\)-algebra generated by the Brownian motion up to time \(t\).

\end{tcolorbox}

\begin{tcolorbox}[enhanced jigsaw, colback=white, rightrule=.15mm, coltitle=black, left=2mm, colframe=quarto-callout-note-color-frame, breakable, arc=.35mm, bottomtitle=1mm, toptitle=1mm, titlerule=0mm, leftrule=.75mm, colbacktitle=quarto-callout-note-color!10!white, title={Definition 1.4 (Adapted Process)}, toprule=.15mm, opacitybacktitle=0.6, bottomrule=.15mm, opacityback=0]

A stochastic process \(\{X_t\}_{t \geq 0}\) is adapted to the filtration
\(\{\mathcal{F}_t\}\) if \(X_t\) is \(\mathcal{F}_t\)-measurable for all
\(t \geq 0\).

\end{tcolorbox}

\begin{tcolorbox}[enhanced jigsaw, colback=white, rightrule=.15mm, coltitle=black, left=2mm, colframe=quarto-callout-note-color-frame, breakable, arc=.35mm, bottomtitle=1mm, toptitle=1mm, titlerule=0mm, leftrule=.75mm, colbacktitle=quarto-callout-note-color!10!white, title={Definition 1.5 (Geometric Brownian Motion)}, toprule=.15mm, opacitybacktitle=0.6, bottomrule=.15mm, opacityback=0]

A process \(\{S_t\}_{t \geq 0}\) follows geometric Brownian motion with
parameters \(\mu \in \mathbb{R}\) and \(\sigma > 0\) if it satisfies the
stochastic differential equation:
\[dS_t = \mu S_t \, dt + \sigma S_t \, dW_t\] with initial condition
\(S_0 > 0\).

\end{tcolorbox}

\begin{tcolorbox}[enhanced jigsaw, colback=white, rightrule=.15mm, coltitle=black, left=2mm, colframe=quarto-callout-tip-color-frame, breakable, arc=.35mm, bottomtitle=1mm, toptitle=1mm, titlerule=0mm, leftrule=.75mm, colbacktitle=quarto-callout-tip-color!10!white, title=\textcolor{quarto-callout-tip-color}{\faLightbulb}\hspace{0.5em}{Remark 1.1}, toprule=.15mm, opacitybacktitle=0.6, bottomrule=.15mm, opacityback=0]

The explicit solution to geometric Brownian motion is:
\[S_t = S_0 \exp\left(\left(\mu - \frac{\sigma^2}{2}\right)t + \sigma W_t\right)\]
This can be verified using Itô's lemma on \(f(t,x) = \ln x\).

\end{tcolorbox}

\hypertarget{portfolio-and-trading-strategy-concepts}{%
\subsection{Portfolio and Trading Strategy
Concepts}\label{portfolio-and-trading-strategy-concepts}}

\begin{tcolorbox}[enhanced jigsaw, colback=white, rightrule=.15mm, coltitle=black, left=2mm, colframe=quarto-callout-note-color-frame, breakable, arc=.35mm, bottomtitle=1mm, toptitle=1mm, titlerule=0mm, leftrule=.75mm, colbacktitle=quarto-callout-note-color!10!white, title={Definition 1.6 (Trading Strategy)}, toprule=.15mm, opacitybacktitle=0.6, bottomrule=.15mm, opacityback=0]

A trading strategy is a pair of adapted processes
\((\phi^0_t, \phi^1_t)\) where:

\begin{itemize}
\tightlist
\item
  \(\phi^0_t\) represents the number of units of the bond held at time
  \(t\)
\item
  \(\phi^1_t\) represents the number of shares of stock held at time
  \(t\)
\end{itemize}

Both processes must be adapted to the filtration \(\{\mathcal{F}_t\}\).

\end{tcolorbox}

\begin{tcolorbox}[enhanced jigsaw, colback=white, rightrule=.15mm, coltitle=black, left=2mm, colframe=quarto-callout-note-color-frame, breakable, arc=.35mm, bottomtitle=1mm, toptitle=1mm, titlerule=0mm, leftrule=.75mm, colbacktitle=quarto-callout-note-color!10!white, title={Definition 1.7 (Portfolio Value)}, toprule=.15mm, opacitybacktitle=0.6, bottomrule=.15mm, opacityback=0]

The value of a portfolio with trading strategy \((\phi^0_t, \phi^1_t)\)
at time \(t\) is: \[V_t = \phi^0_t B_t + \phi^1_t S_t\] where \(B_t\) is
the bond price and \(S_t\) is the stock price.

\end{tcolorbox}

\begin{tcolorbox}[enhanced jigsaw, colback=white, rightrule=.15mm, coltitle=black, left=2mm, colframe=quarto-callout-note-color-frame, breakable, arc=.35mm, bottomtitle=1mm, toptitle=1mm, titlerule=0mm, leftrule=.75mm, colbacktitle=quarto-callout-note-color!10!white, title={Definition 1.8 (Self-Financing Strategy)}, toprule=.15mm, opacitybacktitle=0.6, bottomrule=.15mm, opacityback=0]

A trading strategy \((\phi^0_t, \phi^1_t)\) is self-financing if:
\[dV_t = \phi^0_t dB_t + \phi^1_t dS_t\] This means no money is added or
withdrawn from the portfolio; changes in value come only from price
movements of held assets.

\end{tcolorbox}

\begin{tcolorbox}[enhanced jigsaw, colback=white, rightrule=.15mm, coltitle=black, left=2mm, colframe=quarto-callout-note-color-frame, breakable, arc=.35mm, bottomtitle=1mm, toptitle=1mm, titlerule=0mm, leftrule=.75mm, colbacktitle=quarto-callout-note-color!10!white, title={Definition 1.9 (Arbitrage Opportunity)}, toprule=.15mm, opacitybacktitle=0.6, bottomrule=.15mm, opacityback=0]

An arbitrage opportunity is a self-financing trading strategy with:

\begin{enumerate}
\def\labelenumi{\arabic{enumi}.}
\tightlist
\item
  Initial value \(V_0 = 0\)
\item
  \(\mathbb{P}(V_T \geq 0) = 1\) for some time \(T > 0\)
\item
  \(\mathbb{P}(V_T > 0) > 0\)
\end{enumerate}

\end{tcolorbox}

\begin{tcolorbox}[enhanced jigsaw, colback=white, rightrule=.15mm, coltitle=black, left=2mm, colframe=quarto-callout-note-color-frame, breakable, arc=.35mm, bottomtitle=1mm, toptitle=1mm, titlerule=0mm, leftrule=.75mm, colbacktitle=quarto-callout-note-color!10!white, title={Definition 1.10 (Arbitrage-Free Market)}, toprule=.15mm, opacitybacktitle=0.6, bottomrule=.15mm, opacityback=0]

A market is arbitrage-free if no arbitrage opportunities exist.

\end{tcolorbox}

\hypertarget{martingale-theory-for-finance}{%
\subsection{Martingale Theory for
Finance}\label{martingale-theory-for-finance}}

\begin{tcolorbox}[enhanced jigsaw, colback=white, rightrule=.15mm, coltitle=black, left=2mm, colframe=quarto-callout-note-color-frame, breakable, arc=.35mm, bottomtitle=1mm, toptitle=1mm, titlerule=0mm, leftrule=.75mm, colbacktitle=quarto-callout-note-color!10!white, title={Definition 1.11 (Martingale)}, toprule=.15mm, opacitybacktitle=0.6, bottomrule=.15mm, opacityback=0]

An adapted process \(\{M_t\}_{t \geq 0}\) is a martingale with respect
to filtration \(\{\mathcal{F}_t\}\) and probability measure
\(\mathbb{P}\) if:

\begin{enumerate}
\def\labelenumi{\arabic{enumi}.}
\tightlist
\item
  \(\mathbb{E}[|M_t|] < \infty\) for all \(t \geq 0\)
\item
  \(\mathbb{E}[M_t | \mathcal{F}_s] = M_s\) for all \(0 \leq s \leq t\)
\end{enumerate}

\end{tcolorbox}

\begin{tcolorbox}[enhanced jigsaw, colback=white, rightrule=.15mm, coltitle=black, left=2mm, colframe=quarto-callout-note-color-frame, breakable, arc=.35mm, bottomtitle=1mm, toptitle=1mm, titlerule=0mm, leftrule=.75mm, colbacktitle=quarto-callout-note-color!10!white, title={Definition 1.12 (Equivalent Probability Measures)}, toprule=.15mm, opacitybacktitle=0.6, bottomrule=.15mm, opacityback=0]

Two probability measures \(\mathbb{P}\) and \(\mathbb{Q}\) on
\((\Omega, \mathcal{F})\) are equivalent (written
\(\mathbb{P} \sim \mathbb{Q}\)) if they have the same null sets:
\[\mathbb{P}(A) = 0 \iff \mathbb{Q}(A) = 0 \text{ for all } A \in \mathcal{F}\]

\end{tcolorbox}

\begin{tcolorbox}[enhanced jigsaw, colback=white, rightrule=.15mm, coltitle=black, left=2mm, colframe=quarto-callout-note-color-frame, breakable, arc=.35mm, bottomtitle=1mm, toptitle=1mm, titlerule=0mm, leftrule=.75mm, colbacktitle=quarto-callout-note-color!10!white, title={Definition 1.13 (Radon-Nikodym Derivative)}, toprule=.15mm, opacitybacktitle=0.6, bottomrule=.15mm, opacityback=0]

If \(\mathbb{Q} \ll \mathbb{P}\) (Q is absolutely continuous with
respect to P), then there exists a non-negative
\(\mathcal{F}\)-measurable random variable \(Z\) such that:
\[\mathbb{Q}(A) = \int_A Z \, d\mathbb{P} \text{ for all } A \in \mathcal{F}\]
We write \(Z = \frac{d\mathbb{Q}}{d\mathbb{P}}\) and call \(Z\) the
Radon-Nikodym derivative.

\end{tcolorbox}

\begin{tcolorbox}[enhanced jigsaw, colback=white, rightrule=.15mm, coltitle=black, left=2mm, colframe=quarto-callout-important-color-frame, breakable, arc=.35mm, bottomtitle=1mm, toptitle=1mm, titlerule=0mm, leftrule=.75mm, colbacktitle=quarto-callout-important-color!10!white, title={Theorem 1.1 (Girsanov's Theorem - Statement)}, toprule=.15mm, opacitybacktitle=0.6, bottomrule=.15mm, opacityback=0]

Let \(\theta\) be an adapted process with
\(\int_0^T \theta_s^2 \, ds < \infty\) almost surely. Define:
\[Z_t = \exp\left(-\int_0^t \theta_s \, dW_s - \frac{1}{2}\int_0^t \theta_s^2 \, ds\right)\]

If \(\mathbb{E}[Z_T] = 1\), then:

\begin{enumerate}
\def\labelenumi{\arabic{enumi}.}
\tightlist
\item
  The process \(Z_t\) is a martingale
\item
  The measure \(\mathbb{Q}\) defined by
  \(\frac{d\mathbb{Q}}{d\mathbb{P}} = Z_T\) is a probability measure
  equivalent to \(\mathbb{P}\)
\item
  Under \(\mathbb{Q}\), the process
  \(\tilde{W}_t = W_t + \int_0^t \theta_s \, ds\) is a Brownian motion
\end{enumerate}

\end{tcolorbox}

\begin{tcolorbox}[enhanced jigsaw, colback=white, rightrule=.15mm, coltitle=black, left=2mm, colframe=quarto-callout-note-color-frame, breakable, arc=.35mm, bottomtitle=1mm, toptitle=1mm, titlerule=0mm, leftrule=.75mm, colbacktitle=quarto-callout-note-color!10!white, title={Definition 1.14 (Risk-Neutral Measure)}, toprule=.15mm, opacitybacktitle=0.6, bottomrule=.15mm, opacityback=0]

In a financial market with bond \(B_t = e^{rt}\) and stock following
\[dS_t = \mu S_t \, dt + \sigma S_t \, dW_t\] a probability measure
\(\mathbb{Q}\) equivalent to \(\mathbb{P}\) is called risk-neutral if
the discounted stock price \(e^{-rt}S_t\) is a
\(\mathbb{Q}\)-martingale.

\end{tcolorbox}

\hypertarget{options-and-derivatives}{%
\subsection{Options and Derivatives}\label{options-and-derivatives}}

\begin{tcolorbox}[enhanced jigsaw, colback=white, rightrule=.15mm, coltitle=black, left=2mm, colframe=quarto-callout-note-color-frame, breakable, arc=.35mm, bottomtitle=1mm, toptitle=1mm, titlerule=0mm, leftrule=.75mm, colbacktitle=quarto-callout-note-color!10!white, title={Definition 1.15 (European Option)}, toprule=.15mm, opacitybacktitle=0.6, bottomrule=.15mm, opacityback=0]

A European option is a financial contract that gives the holder the
right (but not obligation) to:

\begin{itemize}
\tightlist
\item
  \textbf{Call option}: Buy an asset at strike price \(K\) at maturity
  time \(T\)
\item
  \textbf{Put option}: Sell an asset at strike price \(K\) at maturity
  time \(T\)
\end{itemize}

\end{tcolorbox}

\begin{tcolorbox}[enhanced jigsaw, colback=white, rightrule=.15mm, coltitle=black, left=2mm, colframe=quarto-callout-note-color-frame, breakable, arc=.35mm, bottomtitle=1mm, toptitle=1mm, titlerule=0mm, leftrule=.75mm, colbacktitle=quarto-callout-note-color!10!white, title={Definition 1.16 (Option Payoff)}, toprule=.15mm, opacitybacktitle=0.6, bottomrule=.15mm, opacityback=0]

The payoff of a European option at maturity \(T\) is:

\begin{itemize}
\tightlist
\item
  \textbf{Call}: \(h(S_T) = (S_T - K)^+ = \max(S_T - K, 0)\)
\item
  \textbf{Put}: \(h(S_T) = (K - S_T)^+ = \max(K - S_T, 0)\)
\end{itemize}

where \(S_T\) is the stock price at maturity.

\end{tcolorbox}

\begin{tcolorbox}[enhanced jigsaw, colback=white, rightrule=.15mm, coltitle=black, left=2mm, colframe=quarto-callout-note-color-frame, breakable, arc=.35mm, bottomtitle=1mm, toptitle=1mm, titlerule=0mm, leftrule=.75mm, colbacktitle=quarto-callout-note-color!10!white, title={Definition 1.17 (Option Price)}, toprule=.15mm, opacitybacktitle=0.6, bottomrule=.15mm, opacityback=0]

The price of an option at time \(t\) is denoted \(V(t, S_t)\) and
depends on the current time \(t\) and current stock price \(S_t\).

\end{tcolorbox}

\begin{tcolorbox}[enhanced jigsaw, colback=white, rightrule=.15mm, coltitle=black, left=2mm, colframe=quarto-callout-note-color-frame, breakable, arc=.35mm, bottomtitle=1mm, toptitle=1mm, titlerule=0mm, leftrule=.75mm, colbacktitle=quarto-callout-note-color!10!white, title={Definition 1.18 (Replicating Portfolio)}, toprule=.15mm, opacitybacktitle=0.6, bottomrule=.15mm, opacityback=0]

A replicating portfolio for an option is a self-financing trading
strategy \((\phi^0_t, \phi^1_t)\) such that the portfolio value at
maturity equals the option payoff:
\[V_T = \phi^0_T B_T + \phi^1_T S_T = h(S_T)\]

\end{tcolorbox}

\hypertarget{fundamental-theorems-of-asset-pricing}{%
\subsection{Fundamental Theorems of Asset
Pricing}\label{fundamental-theorems-of-asset-pricing}}

\begin{tcolorbox}[enhanced jigsaw, colback=white, rightrule=.15mm, coltitle=black, left=2mm, colframe=quarto-callout-important-color-frame, breakable, arc=.35mm, bottomtitle=1mm, toptitle=1mm, titlerule=0mm, leftrule=.75mm, colbacktitle=quarto-callout-important-color!10!white, title={Theorem 1.2 (First Fundamental Theorem of Asset Pricing)}, toprule=.15mm, opacitybacktitle=0.6, bottomrule=.15mm, opacityback=0]

A market is arbitrage-free if and only if there exists a probability
measure \(\mathbb{Q}\) equivalent to \(\mathbb{P}\) such that all
discounted asset prices are \(\mathbb{Q}\)-martingales.

\end{tcolorbox}

\begin{tcolorbox}[enhanced jigsaw, colback=white, rightrule=.15mm, coltitle=black, left=2mm, colframe=quarto-callout-important-color-frame, breakable, arc=.35mm, bottomtitle=1mm, toptitle=1mm, titlerule=0mm, leftrule=.75mm, colbacktitle=quarto-callout-important-color!10!white, title={Theorem 1.3 (Second Fundamental Theorem of Asset Pricing)}, toprule=.15mm, opacitybacktitle=0.6, bottomrule=.15mm, opacityback=0]

An arbitrage-free market is complete if and only if the risk-neutral
measure is unique.

\end{tcolorbox}

\begin{tcolorbox}[enhanced jigsaw, colback=white, rightrule=.15mm, coltitle=black, left=2mm, colframe=quarto-callout-note-color-frame, breakable, arc=.35mm, bottomtitle=1mm, toptitle=1mm, titlerule=0mm, leftrule=.75mm, colbacktitle=quarto-callout-note-color!10!white, title={Definition 1.19 (Complete Market)}, toprule=.15mm, opacitybacktitle=0.6, bottomrule=.15mm, opacityback=0]

A market is complete if every contingent claim (option payoff) can be
replicated by a self-financing trading strategy.

\end{tcolorbox}

\begin{tcolorbox}[enhanced jigsaw, colback=white, rightrule=.15mm, coltitle=black, left=2mm, colframe=quarto-callout-important-color-frame, breakable, arc=.35mm, bottomtitle=1mm, toptitle=1mm, titlerule=0mm, leftrule=.75mm, colbacktitle=quarto-callout-important-color!10!white, title={Theorem 1.4 (Risk-Neutral Valuation Principle)}, toprule=.15mm, opacitybacktitle=0.6, bottomrule=.15mm, opacityback=0]

In an arbitrage-free complete market, the price of any attainable
contingent claim \(h(S_T)\) at time \(t\) is:
\[V(t, S_t) = e^{-r(T-t)} \mathbb{E}_\mathbb{Q}[h(S_T) | \mathcal{F}_t]\]
where \(\mathbb{Q}\) is the unique risk-neutral measure.

\end{tcolorbox}

\hypertarget{part-ii-the-black-scholes-model-setup}{%
\section{Part II: The Black-Scholes Model
Setup}\label{part-ii-the-black-scholes-model-setup}}

\hypertarget{model-specification}{%
\subsection{Model Specification}\label{model-specification}}

We consider a financial market on a probability space
\((\Omega, \mathcal{F}, \mathbb{P})\) with filtration
\(\{\mathcal{F}_t\}_{t \geq 0}\) and two traded assets:

\begin{tcolorbox}[enhanced jigsaw, colback=white, rightrule=.15mm, coltitle=black, left=2mm, colframe=quarto-callout-note-color-frame, breakable, arc=.35mm, bottomtitle=1mm, toptitle=1mm, titlerule=0mm, leftrule=.75mm, colbacktitle=quarto-callout-note-color!10!white, title={Definition 2.1 (Black-Scholes Market Model)}, toprule=.15mm, opacitybacktitle=0.6, bottomrule=.15mm, opacityback=0]

The Black-Scholes market consists of:

\begin{enumerate}
\def\labelenumi{\arabic{enumi}.}
\tightlist
\item
  \textbf{Risk-free bond}: \(dB_t = rB_t \, dt\) with \(B_0 = 1\),
  giving \(B_t = e^{rt}\)
\item
  \textbf{Risky stock}: \(dS_t = \mu S_t \, dt + \sigma S_t \, dW_t\)
  with \(S_0 > 0\)
\end{enumerate}

where:

\begin{itemize}
\tightlist
\item
  \(r > 0\) is the constant risk-free interest rate
\item
  \(\mu \in \mathbb{R}\) is the stock's expected return (drift)
\item
  \(\sigma > 0\) is the stock's volatility
\item
  \(\{W_t\}_{t \geq 0}\) is a standard Brownian motion adapted to
  \(\{\mathcal{F}_t\}\)
\end{itemize}

\end{tcolorbox}

\begin{tcolorbox}[enhanced jigsaw, colback=white, rightrule=.15mm, coltitle=black, left=2mm, colframe=quarto-callout-note-color-frame, breakable, arc=.35mm, bottomtitle=1mm, toptitle=1mm, titlerule=0mm, leftrule=.75mm, colbacktitle=quarto-callout-note-color!10!white, title={Definition 2.2 (European Call Option in Black-Scholes Model)}, toprule=.15mm, opacitybacktitle=0.6, bottomrule=.15mm, opacityback=0]

We consider a European call option with:

\begin{itemize}
\tightlist
\item
  Strike price \(K > 0\)
\item
  Maturity time \(T > 0\)\\
\item
  Payoff at maturity: \(h(S_T) = (S_T - K)^+\)
\end{itemize}

Our goal is to find the option price \(V(t, S_t)\) for
\(0 \leq t \leq T\).

\end{tcolorbox}

\hypertarget{key-properties-of-the-model}{%
\subsection{Key Properties of the
Model}\label{key-properties-of-the-model}}

\begin{tcolorbox}[enhanced jigsaw, colback=white, rightrule=.15mm, coltitle=black, left=2mm, colframe=quarto-callout-important-color-frame, breakable, arc=.35mm, bottomtitle=1mm, toptitle=1mm, titlerule=0mm, leftrule=.75mm, colbacktitle=quarto-callout-important-color!10!white, title={Lemma 2.1 (Stock Price Solution)}, toprule=.15mm, opacitybacktitle=0.6, bottomrule=.15mm, opacityback=0]

The solution to the stock price SDE
\(dS_t = \mu S_t \, dt + \sigma S_t \, dW_t\) is:
\[S_t = S_0 \exp\left(\left(\mu - \frac{\sigma^2}{2}\right)t + \sigma W_t\right)\]

\end{tcolorbox}

\begin{tcolorbox}[enhanced jigsaw, colback=white, rightrule=.15mm, coltitle=black, left=2mm, colframe=quarto-callout-note-color-frame, breakable, arc=.35mm, bottomtitle=1mm, toptitle=1mm, titlerule=0mm, leftrule=.75mm, colbacktitle=quarto-callout-note-color!10!white, title=\textcolor{quarto-callout-note-color}{\faInfo}\hspace{0.5em}{Proof of Lemma 2.1}, toprule=.15mm, opacitybacktitle=0.6, bottomrule=.15mm, opacityback=0]

Let \(Y_t = \ln S_t\). By Itô's lemma with \(f(x) = \ln x\):

\begin{align}
dY_t &= f'(S_t) dS_t + \frac{1}{2}f''(S_t)(dS_t)^2 \\
&= \frac{1}{S_t} dS_t + \frac{1}{2}\left(-\frac{1}{S_t^2}\right)(dS_t)^2 \\
&= \frac{1}{S_t}(\mu S_t \, dt + \sigma S_t \, dW_t) - \frac{1}{2S_t^2}(\sigma S_t)^2 dt \\
&= \mu \, dt + \sigma \, dW_t - \frac{\sigma^2}{2} dt \\
&= \left(\mu - \frac{\sigma^2}{2}\right) dt + \sigma \, dW_t
\end{align}

Integrating from \(0\) to \(t\):
\[Y_t = Y_0 + \left(\mu - \frac{\sigma^2}{2}\right)t + \sigma W_t\]

Since \(Y_t = \ln S_t\) and \(Y_0 = \ln S_0\):
\[\ln S_t = \ln S_0 + \left(\mu - \frac{\sigma^2}{2}\right)t + \sigma W_t\]

Therefore:
\[S_t = S_0 \exp\left(\left(\mu - \frac{\sigma^2}{2}\right)t + \sigma W_t\right)\]

\end{tcolorbox}

\begin{tcolorbox}[enhanced jigsaw, colback=white, rightrule=.15mm, coltitle=black, left=2mm, colframe=quarto-callout-important-color-frame, breakable, arc=.35mm, bottomtitle=1mm, toptitle=1mm, titlerule=0mm, leftrule=.75mm, colbacktitle=quarto-callout-important-color!10!white, title={Corollary 2.1 (Log-Normal Distribution of Stock Price)}, toprule=.15mm, opacitybacktitle=0.6, bottomrule=.15mm, opacityback=0]

Under \(\mathbb{P}\), we have:
\[\ln(S_t/S_0) \sim N\left(\left(\mu - \frac{\sigma^2}{2}\right)t, \sigma^2 t\right)\]

\end{tcolorbox}

\hypertarget{part-iii-proof-via-martingale-approach}{%
\section{Part III: Proof via Martingale
Approach}\label{part-iii-proof-via-martingale-approach}}

\hypertarget{step-1-construction-of-risk-neutral-measure}{%
\subsection{Step 1: Construction of Risk-Neutral
Measure}\label{step-1-construction-of-risk-neutral-measure}}

\begin{tcolorbox}[enhanced jigsaw, colback=white, rightrule=.15mm, coltitle=black, left=2mm, colframe=quarto-callout-important-color-frame, breakable, arc=.35mm, bottomtitle=1mm, toptitle=1mm, titlerule=0mm, leftrule=.75mm, colbacktitle=quarto-callout-important-color!10!white, title={Theorem 3.1 (Existence of Risk-Neutral Measure in Black-Scholes)}, toprule=.15mm, opacitybacktitle=0.6, bottomrule=.15mm, opacityback=0]

There exists a unique probability measure \(\mathbb{Q}\) equivalent to
\(\mathbb{P}\) such that the discounted stock price \(e^{-rt}S_t\) is a
\(\mathbb{Q}\)-martingale.

\end{tcolorbox}

\begin{tcolorbox}[enhanced jigsaw, colback=white, rightrule=.15mm, coltitle=black, left=2mm, colframe=quarto-callout-note-color-frame, breakable, arc=.35mm, bottomtitle=1mm, toptitle=1mm, titlerule=0mm, leftrule=.75mm, colbacktitle=quarto-callout-note-color!10!white, title=\textcolor{quarto-callout-note-color}{\faInfo}\hspace{0.5em}{Proof of Theorem 3.1}, toprule=.15mm, opacitybacktitle=0.6, bottomrule=.15mm, opacityback=0]

\textbf{Step 1}: Define the market price of risk:
\[\theta = \frac{\mu - r}{\sigma}\]

This is the constant that will appear in Girsanov's theorem.

\textbf{Step 2}: Define the Radon-Nikodym density process: \begin{align}
Z_t &= \exp\left(-\theta W_t - \frac{1}{2}\theta^2 t\right) \\
&= \exp\left(-\frac{\mu - r}{\sigma} W_t - \frac{1}{2}\left(\frac{\mu - r}{\sigma}\right)^2 t\right)
\end{align}

\textbf{Step 3}: Verify that \(Z_t\) is a martingale and
\(\mathbb{E}[Z_T] = 1\).

Since \(\theta\) is constant, we can compute:
\[\mathbb{E}[Z_t] = \mathbb{E}\left[\exp\left(-\theta W_t - \frac{1}{2}\theta^2 t\right)\right]\]

Since \(W_t \sim N(0,t)\) under \(\mathbb{P}\): \begin{align}
\mathbb{E}[Z_t] &= \int_{-\infty}^{\infty} \exp\left(-\theta w - \frac{1}{2}\theta^2 t\right) \frac{1}{\sqrt{2\pi t}} \exp\left(-\frac{w^2}{2t}\right) dw \\
&= \exp\left(-\frac{1}{2}\theta^2 t\right) \int_{-\infty}^{\infty} \frac{1}{\sqrt{2\pi t}} \exp\left(-\frac{w^2 + 2\theta tw}{2t}\right) dw \\
&= \exp\left(-\frac{1}{2}\theta^2 t\right) \int_{-\infty}^{\infty} \frac{1}{\sqrt{2\pi t}} \exp\left(-\frac{(w + \theta t)^2 - \theta^2 t^2}{2t}\right) dw \\
&= \exp\left(-\frac{1}{2}\theta^2 t\right) \exp\left(\frac{1}{2}\theta^2 t\right) \int_{-\infty}^{\infty} \frac{1}{\sqrt{2\pi t}} \exp\left(-\frac{(w + \theta t)^2}{2t}\right) dw \\
&= 1
\end{align}

The last integral equals 1 since it's the integral of a normal density.

\textbf{Step 4}: Define \(\mathbb{Q}\) by
\(\frac{d\mathbb{Q}}{d\mathbb{P}} = Z_T\).

\textbf{Step 5}: Apply Girsanov's theorem. Under \(\mathbb{Q}\), the
process: \[\tilde{W}_t = W_t + \int_0^t \theta \, ds = W_t + \theta t\]
is a \(\mathbb{Q}\)-Brownian motion.

\textbf{Step 6}: Transform the stock price dynamics. Under
\(\mathbb{Q}\): \begin{align}
dS_t &= \mu S_t \, dt + \sigma S_t \, dW_t \\
&= \mu S_t \, dt + \sigma S_t (d\tilde{W}_t - \theta \, dt) \\
&= \mu S_t \, dt + \sigma S_t \, d\tilde{W}_t - \sigma S_t \theta \, dt \\
&= (\mu - \sigma\theta) S_t \, dt + \sigma S_t \, d\tilde{W}_t \\
&= \left(\mu - \sigma \cdot \frac{\mu - r}{\sigma}\right) S_t \, dt + \sigma S_t \, d\tilde{W}_t \\
&= r S_t \, dt + \sigma S_t \, d\tilde{W}_t
\end{align}

\textbf{Step 7}: Verify the martingale property. The discounted stock
price is: \[e^{-rt}S_t\]

Its differential is: \begin{align}
d(e^{-rt}S_t) &= -re^{-rt}S_t \, dt + e^{-rt} dS_t \\
&= -re^{-rt}S_t \, dt + e^{-rt}(rS_t \, dt + \sigma S_t \, d\tilde{W}_t) \\
&= e^{-rt}\sigma S_t \, d\tilde{W}_t
\end{align}

Since this has no \(dt\) term, \(e^{-rt}S_t\) is a
\(\mathbb{Q}\)-martingale.

\end{tcolorbox}

\begin{tcolorbox}[enhanced jigsaw, colback=white, rightrule=.15mm, coltitle=black, left=2mm, colframe=quarto-callout-note-color-frame, breakable, arc=.35mm, bottomtitle=1mm, toptitle=1mm, titlerule=0mm, leftrule=.75mm, colbacktitle=quarto-callout-note-color!10!white, title=\textcolor{quarto-callout-note-color}{\faInfo}\hspace{0.5em}{Why Do We Need the Risk-Neutral Measure \(\mathbb{Q}\) at All?}, toprule=.15mm, opacitybacktitle=0.6, bottomrule=.15mm, opacityback=0]

This fundamental question gets to the heart of mathematical finance. The
answer reveals why the risk-neutral measure is not just a mathematical
convenience, but an essential concept for correct option pricing.

\hypertarget{the-fundamental-problem-with-the-physical-measure-mathbbp}{%
\subsubsection{\texorpdfstring{The Fundamental Problem with the Physical
Measure
\(\mathbb{P}\)}{The Fundamental Problem with the Physical Measure \textbackslash mathbb\{P\}}}\label{the-fundamental-problem-with-the-physical-measure-mathbbp}}

Under the physical measure \(\mathbb{P}\), the stock follows:
\[dS_t = \mu S_t dt + \sigma S_t dW_t\]

If we naively tried to price the option directly under \(\mathbb{P}\):
\[V(0, S_0) = e^{-rT} \mathbb{E}_\mathbb{P}[(S_T - K)^+]\]

\textbf{This formula is fundamentally incorrect!} Here's why:

\hypertarget{issue-1-the-discount-rate-problem}{%
\paragraph{Issue 1: The Discount Rate
Problem}\label{issue-1-the-discount-rate-problem}}

The expectation under \(\mathbb{P}\) gives us the expected payoff in the
``real world'' where the stock has expected return \(\mu\). But what
discount rate should we use?

\begin{itemize}
\tightlist
\item
  If \(\mu > r\): The stock is expected to grow faster than the
  risk-free rate
\item
  We might think we should discount at rate \(\mu\), not \(r\)
\item
  But the option's risk profile is completely different from the stock's
  risk profile
\item
  The appropriate discount rate for the option depends on its specific
  risk characteristics
\end{itemize}

\textbf{The key insight}: We don't know what discount rate to use for
the option under \(\mathbb{P}\) because the option's risk premium is
unclear and depends on complex risk preferences.

\hypertarget{issue-2-risk-premium-confusion}{%
\paragraph{Issue 2: Risk Premium
Confusion}\label{issue-2-risk-premium-confusion}}

Under the physical measure \(\mathbb{P}\): - The stock earns expected
return \(\mu\) - The bond earns certain return \(r\) - The difference
\(\mu - r\) is the equity risk premium

But the option has a completely different risk profile than the stock!
Its appropriate risk premium is not obvious and would require knowledge
of: - Investor risk preferences - The option's correlation with market
factors - The option's beta relative to various risk factors

\hypertarget{why-the-risk-neutral-measure-mathbbq-solves-this-elegantly}{%
\subsubsection{\texorpdfstring{Why the Risk-Neutral Measure
\(\mathbb{Q}\) Solves This
Elegantly}{Why the Risk-Neutral Measure \textbackslash mathbb\{Q\} Solves This Elegantly}}\label{why-the-risk-neutral-measure-mathbbq-solves-this-elegantly}}

The brilliant insight is to change to a probability measure where
\textbf{all assets have the same expected return} equal to the risk-free
rate.

\hypertarget{under-mathbbq}{%
\paragraph{\texorpdfstring{Under
\(\mathbb{Q}\):}{Under \textbackslash mathbb\{Q\}:}}\label{under-mathbbq}}

\begin{itemize}
\tightlist
\item
  Stock expected return: \(r\)
\item
  Bond expected return: \(r\)
\item
  \textbf{Any derivative's expected return: \(r\)}
\end{itemize}

This means we can discount everything at the risk-free rate \(r\)!

\[V(0, S_0) = e^{-rT} \mathbb{E}_\mathbb{Q}[(S_T - K)^+]\]

Now the formula is correct because: 1. The expected return of the option
under \(\mathbb{Q}\) is exactly \(r\) 2. So we discount at rate \(r\) 3.
The mathematics works out perfectly 4. No risk premiums need to be
determined

\hypertarget{the-economic-intuition}{%
\subsubsection{The Economic Intuition}\label{the-economic-intuition}}

Think of the two measures this way:

\textbf{Physical measure \(\mathbb{P}\)}: \emph{``What will actually
happen in the real world?''} - Stocks have risk premiums reflecting
investor risk aversion - Different assets have different expected
returns - Discount rates are asset-specific and difficult to determine -
Requires knowledge of risk preferences and market prices of risk

\textbf{Risk-neutral measure \(\mathbb{Q}\)}: \emph{``What would happen
in a hypothetical world where all investors are risk-neutral?''} - All
assets earn the risk-free rate in expectation - No risk premiums exist -
Everything can be discounted at the same rate \(r\) - Completely
bypasses the need to know risk preferences

\hypertarget{the-arbitrage-connection}{%
\subsubsection{The Arbitrage
Connection}\label{the-arbitrage-connection}}

The risk-neutral measure exists precisely because there are no arbitrage
opportunities. Here's the fundamental logic:

\begin{enumerate}
\def\labelenumi{\arabic{enumi}.}
\tightlist
\item
  \textbf{No arbitrage} ⟺ \textbf{Risk-neutral measure exists} (First
  Fundamental Theorem of Asset Pricing)
\item
  If we can replicate the option with a portfolio of stock and bonds,
  then the option price must equal the portfolio value (no arbitrage
  principle)
\item
  The replicating portfolio approach automatically gives us the
  risk-neutral valuation
\item
  The \(\mathbb{Q}\) measure is the unique measure that makes this work
\end{enumerate}

\hypertarget{a-concrete-example}{%
\subsubsection{A Concrete Example}\label{a-concrete-example}}

Consider a simple one-period binomial model:

\textbf{Setup:} - Stock price: \(S_0 = 100\) - Up move: \(S_1 = 120\)
with probability \(p = 0.6\) under \(\mathbb{P}\) - Down move:
\(S_1 = 80\) with probability \(1-p = 0.4\) under \(\mathbb{P}\) -
Risk-free rate: \(r = 5\%\)

\textbf{Under \(\mathbb{P}\)}: Expected stock return is
\(0.6 \times 20\% + 0.4 \times (-20\%) = 4\%\)

\textbf{For a call option with \(K = 100\)}: - Payoff if stock goes up:
\(\max(120-100, 0) = 20\) - Payoff if stock goes down:
\(\max(80-100, 0) = 0\)

\hypertarget{wrong-approach-using-mathbbp}{%
\paragraph{\texorpdfstring{Wrong Approach (using
\(\mathbb{P}\)):}{Wrong Approach (using \textbackslash mathbb\{P\}):}}\label{wrong-approach-using-mathbbp}}

\[V_0 = \frac{0.6 \times 20 + 0.4 \times 0}{1.05} = \frac{12}{1.05} = 11.43\]

This is incorrect because we're using the wrong probabilities for
discounting at the risk-free rate.

\hypertarget{correct-approach-using-mathbbq}{%
\paragraph{\texorpdfstring{Correct Approach (using
\(\mathbb{Q}\)):}{Correct Approach (using \textbackslash mathbb\{Q\}):}}\label{correct-approach-using-mathbbq}}

First, find the risk-neutral probabilities. Under \(\mathbb{Q}\), the
stock must have expected return equal to the risk-free rate \(r = 5\%\):

\[q \times 120 + (1-q) \times 80 = 100 \times 1.05 = 105\]
\[40q + 80 = 105\] \[q = 0.625\]

Now we can price correctly:
\[V_0 = \frac{0.625 \times 20 + 0.375 \times 0}{1.05} = \frac{12.5}{1.05} = 11.90\]

The risk-neutral approach gives the unique arbitrage-free price!

\hypertarget{verification-by-replication}{%
\paragraph{Verification by
Replication:}\label{verification-by-replication}}

We can verify this is correct by constructing a replicating portfolio: -
Buy \(\Delta\) shares of stock - Invest \(B\) in bonds

Portfolio value in up state: \(120\Delta + 1.05B = 20\) Portfolio value
in down state: \(80\Delta + 1.05B = 0\)

Solving: \(\Delta = 0.5\), \(B = -38.10\)

Initial portfolio cost: \(100 \times 0.5 - 38.10 = 11.90\) ✓

\hypertarget{why-this-matters-for-black-scholes}{%
\subsubsection{Why This Matters for
Black-Scholes}\label{why-this-matters-for-black-scholes}}

In the Black-Scholes framework:

\begin{enumerate}
\def\labelenumi{\arabic{enumi}.}
\tightlist
\item
  \textbf{Physical measure}: Stock has drift \(\mu\), but option pricing
  would require determining the option's risk premium
\item
  \textbf{Risk-neutral measure}: Stock has drift \(r\), and option
  pricing becomes a pure expectation calculation
\item
  \textbf{Girsanov's theorem}: Provides the mathematical machinery to
  change from \(\mathbb{P}\) to \(\mathbb{Q}\)
\item
  \textbf{Hedging connection}: The \(\mathbb{Q}\) measure emerges
  naturally from the delta-hedging argument
\end{enumerate}

\hypertarget{summary-why-we-need-mathbbq}{%
\subsubsection{\texorpdfstring{Summary: Why We Need
\(\mathbb{Q}\)}{Summary: Why We Need \textbackslash mathbb\{Q\}}}\label{summary-why-we-need-mathbbq}}

The risk-neutral measure is essential because it:

\begin{enumerate}
\def\labelenumi{\arabic{enumi}.}
\tightlist
\item
  \textbf{Eliminates risk premium confusion} - all assets earn rate
  \(r\) in expectation
\item
  \textbf{Provides the correct discount rate} - always the risk-free
  rate \(r\)
\item
  \textbf{Gives the unique arbitrage-free approach} - guaranteed by
  fundamental theorems
\item
  \textbf{Connects to replication strategies} - matches the
  hedging-based derivation perfectly
\item
  \textbf{Makes pricing computationally tractable} - turns complex
  pricing into expectation calculations
\item
  \textbf{Bypasses investor preferences} - no need to know risk aversion
  parameters
\item
  \textbf{Ensures market completeness} - works for any derivative in a
  complete market
\end{enumerate}

\textbf{The key insight}: The risk-neutral measure is not about what
will actually happen in reality - it's about finding the unique
arbitrage-free price in a mathematically elegant and practically
implementable way. It transforms the complex problem of determining
risk-adjusted discount rates into the simpler problem of computing
expectations under an artificial but mathematically convenient
probability measure.

\end{tcolorbox}

\hypertarget{step-2-stock-price-under-risk-neutral-measure}{%
\subsection{Step 2: Stock Price Under Risk-Neutral
Measure}\label{step-2-stock-price-under-risk-neutral-measure}}

\begin{tcolorbox}[enhanced jigsaw, colback=white, rightrule=.15mm, coltitle=black, left=2mm, colframe=quarto-callout-important-color-frame, breakable, arc=.35mm, bottomtitle=1mm, toptitle=1mm, titlerule=0mm, leftrule=.75mm, colbacktitle=quarto-callout-important-color!10!white, title={Lemma 3.1 (Stock Price Solution Under \(\mathbb{Q}\))}, toprule=.15mm, opacitybacktitle=0.6, bottomrule=.15mm, opacityback=0]

Under the risk-neutral measure \(\mathbb{Q}\), the stock price has the
explicit form:
\[S_t = S_0 \exp\left(\left(r - \frac{\sigma^2}{2}\right)t + \sigma \tilde{W}_t\right)\]
where \(\tilde{W}_t\) is a \(\mathbb{Q}\)-Brownian motion.

\end{tcolorbox}

\begin{tcolorbox}[enhanced jigsaw, colback=white, rightrule=.15mm, coltitle=black, left=2mm, colframe=quarto-callout-note-color-frame, breakable, arc=.35mm, bottomtitle=1mm, toptitle=1mm, titlerule=0mm, leftrule=.75mm, colbacktitle=quarto-callout-note-color!10!white, title=\textcolor{quarto-callout-note-color}{\faInfo}\hspace{0.5em}{Proof of Lemma 3.1}, toprule=.15mm, opacitybacktitle=0.6, bottomrule=.15mm, opacityback=0]

Under \(\mathbb{Q}\), the stock follows:
\[dS_t = rS_t \, dt + \sigma S_t \, d\tilde{W}_t\]

Using the same technique as in the original measure, let
\(Y_t = \ln S_t\): \begin{align}
dY_t &= \frac{1}{S_t} dS_t - \frac{1}{2S_t^2}(dS_t)^2 \\
&= \frac{1}{S_t}(rS_t \, dt + \sigma S_t \, d\tilde{W}_t) - \frac{1}{2S_t^2}(\sigma S_t)^2 dt \\
&= r \, dt + \sigma \, d\tilde{W}_t - \frac{\sigma^2}{2} dt \\
&= \left(r - \frac{\sigma^2}{2}\right) dt + \sigma \, d\tilde{W}_t
\end{align}

Integrating:
\[Y_t = Y_0 + \left(r - \frac{\sigma^2}{2}\right)t + \sigma \tilde{W}_t\]

Therefore:
\[S_t = S_0 \exp\left(\left(r - \frac{\sigma^2}{2}\right)t + \sigma \tilde{W}_t\right)\]

\end{tcolorbox}

\hypertarget{step-3-risk-neutral-valuation}{%
\subsection{Step 3: Risk-Neutral
Valuation}\label{step-3-risk-neutral-valuation}}

\begin{tcolorbox}[enhanced jigsaw, colback=white, rightrule=.15mm, coltitle=black, left=2mm, colframe=quarto-callout-important-color-frame, breakable, arc=.35mm, bottomtitle=1mm, toptitle=1mm, titlerule=0mm, leftrule=.75mm, colbacktitle=quarto-callout-important-color!10!white, title={Theorem 3.2 (Risk-Neutral Option Pricing)}, toprule=.15mm, opacitybacktitle=0.6, bottomrule=.15mm, opacityback=0]

The price of the European call option at time \(t\) is:
\[V(t, S_t) = e^{-r(T-t)} \mathbb{E}_\mathbb{Q}[(S_T - K)^+ | \mathcal{F}_t]\]

\end{tcolorbox}

\begin{tcolorbox}[enhanced jigsaw, colback=white, rightrule=.15mm, coltitle=black, left=2mm, colframe=quarto-callout-note-color-frame, breakable, arc=.35mm, bottomtitle=1mm, toptitle=1mm, titlerule=0mm, leftrule=.75mm, colbacktitle=quarto-callout-note-color!10!white, title=\textcolor{quarto-callout-note-color}{\faInfo}\hspace{0.5em}{Proof of Theorem 3.2}, toprule=.15mm, opacitybacktitle=0.6, bottomrule=.15mm, opacityback=0]

This follows directly from the risk-neutral valuation principle. In an
arbitrage-free complete market, the option price equals the discounted
expected payoff under the risk-neutral measure.

\end{tcolorbox}

\hypertarget{step-4-computing-the-expectation}{%
\subsection{Step 4: Computing the
Expectation}\label{step-4-computing-the-expectation}}

\begin{tcolorbox}[enhanced jigsaw, colback=white, rightrule=.15mm, coltitle=black, left=2mm, colframe=quarto-callout-important-color-frame, breakable, arc=.35mm, bottomtitle=1mm, toptitle=1mm, titlerule=0mm, leftrule=.75mm, colbacktitle=quarto-callout-important-color!10!white, title={Theorem 3.3 (Black-Scholes Formula via Martingale Method)}, toprule=.15mm, opacitybacktitle=0.6, bottomrule=.15mm, opacityback=0]

The European call option price is:
\[V(t, S_t) = S_t \Phi(d_1) - K e^{-r(T-t)} \Phi(d_2)\] where:
\begin{align}
d_1 &= \frac{\ln(S_t/K) + (r + \sigma^2/2)(T-t)}{\sigma\sqrt{T-t}} \\
d_2 &= d_1 - \sigma\sqrt{T-t} = \frac{\ln(S_t/K) + (r - \sigma^2/2)(T-t)}{\sigma\sqrt{T-t}}
\end{align} and \(\Phi\) is the cumulative distribution function of the
standard normal distribution.

\end{tcolorbox}

\begin{tcolorbox}[enhanced jigsaw, colback=white, rightrule=.15mm, coltitle=black, left=2mm, colframe=quarto-callout-note-color-frame, breakable, arc=.35mm, bottomtitle=1mm, toptitle=1mm, titlerule=0mm, leftrule=.75mm, colbacktitle=quarto-callout-note-color!10!white, title=\textcolor{quarto-callout-note-color}{\faInfo}\hspace{0.5em}{Proof of Theorem 3.3}, toprule=.15mm, opacitybacktitle=0.6, bottomrule=.15mm, opacityback=0]

\textbf{Step 1}: Express the stock price at maturity. Under
\(\mathbb{Q}\), using the strong Markov property:
\[S_T = S_t \exp\left(\left(r - \frac{\sigma^2}{2}\right)(T-t) + \sigma (\tilde{W}_T - \tilde{W}_t)\right)\]

Since \(\tilde{W}_T - \tilde{W}_t \sim N(0, T-t)\) under \(\mathbb{Q}\),
let \(Z \sim N(0,1)\). Then:
\[S_T = S_t \exp\left(\left(r - \frac{\sigma^2}{2}\right)(T-t) + \sigma\sqrt{T-t} \cdot Z\right)\]

\textbf{Step 2}: Set up the expectation. \begin{align}
V(t, S_t) &= e^{-r(T-t)} \mathbb{E}_\mathbb{Q}[(S_T - K)^+ | \mathcal{F}_t] \\
&= e^{-r(T-t)} \mathbb{E}_\mathbb{Q}\left[\left(S_t e^{(r-\sigma^2/2)(T-t) + \sigma\sqrt{T-t} Z} - K\right)^+\right]
\end{align}

\textbf{Step 3}: Find the exercise region. The option is exercised when
\(S_T > K\), i.e., when:
\[S_t e^{(r-\sigma^2/2)(T-t) + \sigma\sqrt{T-t} Z} > K\]

Taking logarithms: \begin{align}
&(r-\sigma^2/2)(T-t) + \sigma\sqrt{T-t} Z > \ln(K/S_t) \\
&Z > \frac{\ln(K/S_t) - (r-\sigma^2/2)(T-t)}{\sigma\sqrt{T-t}} = -d_2
\end{align}

\textbf{Step 4}: Evaluate the integral. \begin{align}
V(t, S_t) &= e^{-r(T-t)} \int_{-d_2}^{\infty} \left(S_t e^{(r-\sigma^2/2)(T-t) + \sigma\sqrt{T-t} z} - K\right) \\
&\quad \times \frac{1}{\sqrt{2\pi}} e^{-z^2/2} dz \\
&= e^{-r(T-t)} S_t e^{(r-\sigma^2/2)(T-t)} \int_{-d_2}^{\infty} e^{\sigma\sqrt{T-t} z} \frac{1}{\sqrt{2\pi}} e^{-z^2/2} dz \\
&\quad - e^{-r(T-t)} K \int_{-d_2}^{\infty} \frac{1}{\sqrt{2\pi}} e^{-z^2/2} dz
\end{align}

\textbf{Step 5}: Evaluate the first integral. For the first integral,
complete the square in the exponent:
\[\sigma\sqrt{T-t} z - \frac{z^2}{2} = -\frac{1}{2}(z - \sigma\sqrt{T-t})^2 + \frac{\sigma^2(T-t)}{2}\]

Therefore: \begin{align}
&\int_{-d_2}^{\infty} e^{\sigma\sqrt{T-t} z} \frac{1}{\sqrt{2\pi}} e^{-z^2/2} dz \\
&= e^{\sigma^2(T-t)/2} \int_{-d_2}^{\infty} \frac{1}{\sqrt{2\pi}} e^{-(z-\sigma\sqrt{T-t})^2/2} dz \\
&= e^{\sigma^2(T-t)/2} \int_{-d_2-\sigma\sqrt{T-t}}^{\infty} \frac{1}{\sqrt{2\pi}} e^{-u^2/2} du \\
&= e^{\sigma^2(T-t)/2} \Phi(d_2 + \sigma\sqrt{T-t}) \\
&= e^{\sigma^2(T-t)/2} \Phi(d_1)
\end{align}

where we used the substitution \(u = z - \sigma\sqrt{T-t}\) and the fact
that: \[d_2 + \sigma\sqrt{T-t} = d_1\]

\textbf{Step 6}: Evaluate the second integral.
\[\int_{-d_2}^{\infty} \frac{1}{\sqrt{2\pi}} e^{-z^2/2} dz = \Phi(d_2)\]

\textbf{Step 7}: Combine the results. \begin{align}
V(t, S_t) &= e^{-r(T-t)} S_t e^{(r-\sigma^2/2)(T-t)} e^{\sigma^2(T-t)/2} \Phi(d_1) - e^{-r(T-t)} K \Phi(d_2) \\
&= S_t e^{-r(T-t)} e^{(r-\sigma^2/2)(T-t)} e^{\sigma^2(T-t)/2} \Phi(d_1) - K e^{-r(T-t)} \Phi(d_2) \\
&= S_t e^{-r(T-t) + r(T-t) - \sigma^2(T-t)/2 + \sigma^2(T-t)/2} \Phi(d_1) - K e^{-r(T-t)} \Phi(d_2) \\
&= S_t \Phi(d_1) - K e^{-r(T-t)} \Phi(d_2)
\end{align}

\end{tcolorbox}

\hypertarget{part-iv-proof-via-pde-approach}{%
\section{Part IV: Proof via PDE
Approach}\label{part-iv-proof-via-pde-approach}}

\hypertarget{step-1-delta-hedging-strategy}{%
\subsection{Step 1: Delta Hedging
Strategy}\label{step-1-delta-hedging-strategy}}

\begin{tcolorbox}[enhanced jigsaw, colback=white, rightrule=.15mm, coltitle=black, left=2mm, colframe=quarto-callout-note-color-frame, breakable, arc=.35mm, bottomtitle=1mm, toptitle=1mm, titlerule=0mm, leftrule=.75mm, colbacktitle=quarto-callout-note-color!10!white, title={Definition 4.1 (Delta Hedging Portfolio)}, toprule=.15mm, opacitybacktitle=0.6, bottomrule=.15mm, opacityback=0]

Consider a portfolio \(\Pi_t\) consisting of:

\begin{itemize}
\tightlist
\item
  Long position: 1 unit of the option with value \(V(t, S_t)\)
\item
  Short position: \(\Delta_t\) shares of the stock
\item
  Cash position: The remaining value invested in bonds
\end{itemize}

The portfolio value is: \(\Pi_t = V(t, S_t) - \Delta_t S_t\)

\end{tcolorbox}

\begin{tcolorbox}[enhanced jigsaw, colback=white, rightrule=.15mm, coltitle=black, left=2mm, colframe=quarto-callout-note-color-frame, breakable, arc=.35mm, bottomtitle=1mm, toptitle=1mm, titlerule=0mm, leftrule=.75mm, colbacktitle=quarto-callout-note-color!10!white, title={Definition 4.2 (Delta)}, toprule=.15mm, opacitybacktitle=0.6, bottomrule=.15mm, opacityback=0]

The delta \(\Delta_t\) represents the hedge ratio - the number of shares
of stock to short for each option held. We will determine \(\Delta_t\)
to make the portfolio risk-free.

\end{tcolorbox}

\hypertarget{step-2-application-of-ituxf4s-lemma-to-option-value}{%
\subsection{Step 2: Application of Itô's Lemma to Option
Value}\label{step-2-application-of-ituxf4s-lemma-to-option-value}}

\begin{tcolorbox}[enhanced jigsaw, colback=white, rightrule=.15mm, coltitle=black, left=2mm, colframe=quarto-callout-important-color-frame, breakable, arc=.35mm, bottomtitle=1mm, toptitle=1mm, titlerule=0mm, leftrule=.75mm, colbacktitle=quarto-callout-important-color!10!white, title={Lemma 4.1 (Stochastic Differential of Option Value)}, toprule=.15mm, opacitybacktitle=0.6, bottomrule=.15mm, opacityback=0]

Assume the option price has the form \(V(t, S_t)\) where
\(V \in C^{1,2}([0,T] \times (0,\infty))\). Then by Itô's lemma:
\begin{align}
dV &= \frac{\partial V}{\partial t} dt + \frac{\partial V}{\partial S} dS_t + \frac{1}{2}\frac{\partial^2 V}{\partial S^2} (dS_t)^2 \\
&= \left[\frac{\partial V}{\partial t} + \mu S_t \frac{\partial V}{\partial S} + \frac{1}{2}\sigma^2 S_t^2 \frac{\partial^2 V}{\partial S^2}\right] dt + \sigma S_t \frac{\partial V}{\partial S} dW_t
\end{align}

\end{tcolorbox}

\begin{tcolorbox}[enhanced jigsaw, colback=white, rightrule=.15mm, coltitle=black, left=2mm, colframe=quarto-callout-note-color-frame, breakable, arc=.35mm, bottomtitle=1mm, toptitle=1mm, titlerule=0mm, leftrule=.75mm, colbacktitle=quarto-callout-note-color!10!white, title=\textcolor{quarto-callout-note-color}{\faInfo}\hspace{0.5em}{Proof of Lemma 4.1}, toprule=.15mm, opacitybacktitle=0.6, bottomrule=.15mm, opacityback=0]

Since \(dS_t = \mu S_t \, dt + \sigma S_t \, dW_t\), we have:
\((dS_t)^2 = (\mu S_t \, dt + \sigma S_t \, dW_t)^2 = \sigma^2 S_t^2 (dW_t)^2 = \sigma^2 S_t^2 dt\)

where we used the fact that \((dW_t)^2 = dt\), \(dt \cdot dW_t = 0\),
and \((dt)^2 = 0\).

Applying Itô's lemma: \begin{align}
dV &= \frac{\partial V}{\partial t} dt + \frac{\partial V}{\partial S} dS_t + \frac{1}{2}\frac{\partial^2 V}{\partial S^2} (dS_t)^2 \\
&= \frac{\partial V}{\partial t} dt + \frac{\partial V}{\partial S} (\mu S_t \, dt + \sigma S_t \, dW_t) + \frac{1}{2}\frac{\partial^2 V}{\partial S^2} \sigma^2 S_t^2 dt \\
&= \left[\frac{\partial V}{\partial t} + \mu S_t \frac{\partial V}{\partial S} + \frac{1}{2}\sigma^2 S_t^2 \frac{\partial^2 V}{\partial S^2}\right] dt + \sigma S_t \frac{\partial V}{\partial S} dW_t
\end{align}

\end{tcolorbox}

\hypertarget{step-3-portfolio-dynamics}{%
\subsection{Step 3: Portfolio
Dynamics}\label{step-3-portfolio-dynamics}}

\begin{tcolorbox}[enhanced jigsaw, colback=white, rightrule=.15mm, coltitle=black, left=2mm, colframe=quarto-callout-important-color-frame, breakable, arc=.35mm, bottomtitle=1mm, toptitle=1mm, titlerule=0mm, leftrule=.75mm, colbacktitle=quarto-callout-important-color!10!white, title={Lemma 4.2 (Portfolio Value Dynamics)}, toprule=.15mm, opacitybacktitle=0.6, bottomrule=.15mm, opacityback=0]

The change in portfolio value is: \begin{align}
d\Pi_t &= dV - \Delta_t dS_t \\
&= \left[\frac{\partial V}{\partial t} + \mu S_t \left(\frac{\partial V}{\partial S} - \Delta_t\right) + \frac{1}{2}\sigma^2 S_t^2 \frac{\partial^2 V}{\partial S^2}\right] dt \\
&\quad + \sigma S_t \left(\frac{\partial V}{\partial S} - \Delta_t\right) dW_t
\end{align}

\end{tcolorbox}

\begin{tcolorbox}[enhanced jigsaw, colback=white, rightrule=.15mm, coltitle=black, left=2mm, colframe=quarto-callout-note-color-frame, breakable, arc=.35mm, bottomtitle=1mm, toptitle=1mm, titlerule=0mm, leftrule=.75mm, colbacktitle=quarto-callout-note-color!10!white, title=\textcolor{quarto-callout-note-color}{\faInfo}\hspace{0.5em}{Proof of Lemma 4.2}, toprule=.15mm, opacitybacktitle=0.6, bottomrule=.15mm, opacityback=0]

We have: \begin{align}
d\Pi_t &= dV - \Delta_t dS_t \\
&= \left[\frac{\partial V}{\partial t} + \mu S_t \frac{\partial V}{\partial S} + \frac{1}{2}\sigma^2 S_t^2 \frac{\partial^2 V}{\partial S^2}\right] dt + \sigma S_t \frac{\partial V}{\partial S} dW_t \\
&\quad - \Delta_t (\mu S_t dt + \sigma S_t dW_t) \\
&= \left[\frac{\partial V}{\partial t} + \mu S_t \frac{\partial V}{\partial S} + \frac{1}{2}\sigma^2 S_t^2 \frac{\partial^2 V}{\partial S^2} - \Delta_t \mu S_t\right] dt \\
&\quad + \sigma S_t \left(\frac{\partial V}{\partial S} - \Delta_t\right) dW_t \\
&= \left[\frac{\partial V}{\partial t} + \mu S_t \left(\frac{\partial V}{\partial S} - \Delta_t\right) + \frac{1}{2}\sigma^2 S_t^2 \frac{\partial^2 V}{\partial S^2}\right] dt \\
&\quad + \sigma S_t \left(\frac{\partial V}{\partial S} - \Delta_t\right) dW_t
\end{align}

\end{tcolorbox}

\hypertarget{step-4-eliminating-risk}{%
\subsection{Step 4: Eliminating Risk}\label{step-4-eliminating-risk}}

\begin{tcolorbox}[enhanced jigsaw, colback=white, rightrule=.15mm, coltitle=black, left=2mm, colframe=quarto-callout-note-color-frame, breakable, arc=.35mm, bottomtitle=1mm, toptitle=1mm, titlerule=0mm, leftrule=.75mm, colbacktitle=quarto-callout-note-color!10!white, title={Definition 4.3 (Delta-Neutral Portfolio)}, toprule=.15mm, opacitybacktitle=0.6, bottomrule=.15mm, opacityback=0]

To eliminate the stochastic component, we choose:
\(\Delta_t = \frac{\partial V}{\partial S}\)

This choice makes the portfolio instantaneously risk-free.

\end{tcolorbox}

\begin{tcolorbox}[enhanced jigsaw, colback=white, rightrule=.15mm, coltitle=black, left=2mm, colframe=quarto-callout-important-color-frame, breakable, arc=.35mm, bottomtitle=1mm, toptitle=1mm, titlerule=0mm, leftrule=.75mm, colbacktitle=quarto-callout-important-color!10!white, title={Lemma 4.3 (Risk-Free Portfolio Dynamics)}, toprule=.15mm, opacitybacktitle=0.6, bottomrule=.15mm, opacityback=0]

With \(\Delta_t = \frac{\partial V}{\partial S}\), the portfolio
dynamics become:
\(d\Pi_t = \left[\frac{\partial V}{\partial t} + \frac{1}{2}\sigma^2 S_t^2 \frac{\partial^2 V}{\partial S^2}\right] dt\)

\end{tcolorbox}

\hypertarget{step-5-no-arbitrage-condition}{%
\subsection{Step 5: No-Arbitrage
Condition}\label{step-5-no-arbitrage-condition}}

\begin{tcolorbox}[enhanced jigsaw, colback=white, rightrule=.15mm, coltitle=black, left=2mm, colframe=quarto-callout-important-color-frame, breakable, arc=.35mm, bottomtitle=1mm, toptitle=1mm, titlerule=0mm, leftrule=.75mm, colbacktitle=quarto-callout-important-color!10!white, title={Theorem 4.1 (Risk-Free Rate Condition)}, toprule=.15mm, opacitybacktitle=0.6, bottomrule=.15mm, opacityback=0]

Since the portfolio is risk-free, it must earn the risk-free rate to
prevent arbitrage: \(d\Pi_t = r\Pi_t dt\)

Substituting
\(\Pi_t = V - \Delta_t S_t = V - \frac{\partial V}{\partial S} S_t\):
\(d\Pi_t = r\left(V - \frac{\partial V}{\partial S} S_t\right) dt\)

\end{tcolorbox}

\begin{tcolorbox}[enhanced jigsaw, colback=white, rightrule=.15mm, coltitle=black, left=2mm, colframe=quarto-callout-note-color-frame, breakable, arc=.35mm, bottomtitle=1mm, toptitle=1mm, titlerule=0mm, leftrule=.75mm, colbacktitle=quarto-callout-note-color!10!white, title=\textcolor{quarto-callout-note-color}{\faInfo}\hspace{0.5em}{Proof of Theorem 4.1}, toprule=.15mm, opacitybacktitle=0.6, bottomrule=.15mm, opacityback=0]

If a risk-free portfolio earned more than the risk-free rate, we could
borrow at rate \(r\), invest in the portfolio, and earn a risk-free
profit (arbitrage). If it earned less, we could short the portfolio,
lend at rate \(r\), and again earn risk-free profit.

\end{tcolorbox}

\hypertarget{step-6-deriving-the-black-scholes-pde}{%
\subsection{Step 6: Deriving the Black-Scholes
PDE}\label{step-6-deriving-the-black-scholes-pde}}

\begin{tcolorbox}[enhanced jigsaw, colback=white, rightrule=.15mm, coltitle=black, left=2mm, colframe=quarto-callout-important-color-frame, breakable, arc=.35mm, bottomtitle=1mm, toptitle=1mm, titlerule=0mm, leftrule=.75mm, colbacktitle=quarto-callout-important-color!10!white, title={Theorem 4.2 (Black-Scholes Partial Differential Equation)}, toprule=.15mm, opacitybacktitle=0.6, bottomrule=.15mm, opacityback=0]

The option price \(V(t,S)\) satisfies the Black-Scholes PDE:
\(\frac{\partial V}{\partial t} + rS \frac{\partial V}{\partial S} + \frac{1}{2}\sigma^2 S^2 \frac{\partial^2 V}{\partial S^2} - rV = 0\)

with terminal condition: \(V(T,S) = (S-K)^+\)

\end{tcolorbox}

\begin{tcolorbox}[enhanced jigsaw, colback=white, rightrule=.15mm, coltitle=black, left=2mm, colframe=quarto-callout-note-color-frame, breakable, arc=.35mm, bottomtitle=1mm, toptitle=1mm, titlerule=0mm, leftrule=.75mm, colbacktitle=quarto-callout-note-color!10!white, title=\textcolor{quarto-callout-note-color}{\faInfo}\hspace{0.5em}{Proof of Theorem 4.2}, toprule=.15mm, opacitybacktitle=0.6, bottomrule=.15mm, opacityback=0]

Equating the two expressions for \(d\Pi_t\):
\(\frac{\partial V}{\partial t} + \frac{1}{2}\sigma^2 S_t^2 \frac{\partial^2 V}{\partial S^2} = r\left(V - \frac{\partial V}{\partial S} S_t\right)\)

Rearranging:
\(\frac{\partial V}{\partial t} + \frac{1}{2}\sigma^2 S_t^2 \frac{\partial^2 V}{\partial S^2} = rV - rS_t \frac{\partial V}{\partial S}\)

Therefore:
\(\frac{\partial V}{\partial t} + rS_t \frac{\partial V}{\partial S} + \frac{1}{2}\sigma^2 S_t^2 \frac{\partial^2 V}{\partial S^2} - rV = 0\)

The terminal condition comes from the option payoff at maturity.

\end{tcolorbox}

\hypertarget{step-7-solving-the-pde}{%
\subsection{Step 7: Solving the PDE}\label{step-7-solving-the-pde}}

To solve this PDE, we use a change of variables to transform it into the
heat equation.

\begin{tcolorbox}[enhanced jigsaw, colback=white, rightrule=.15mm, coltitle=black, left=2mm, colframe=quarto-callout-note-color-frame, breakable, arc=.35mm, bottomtitle=1mm, toptitle=1mm, titlerule=0mm, leftrule=.75mm, colbacktitle=quarto-callout-note-color!10!white, title={Definition 4.4 (Variable Transformation)}, toprule=.15mm, opacitybacktitle=0.6, bottomrule=.15mm, opacityback=0]

Let: \begin{align}
\tau &= T - t \quad \text{(time to maturity)} \\
x &= \ln(S/K) \quad \text{(log-moneyness)} \\
u(\tau, x) &= e^{r\tau} \frac{V(T-\tau, Ke^x)}{K}
\end{align}

\end{tcolorbox}

\begin{tcolorbox}[enhanced jigsaw, colback=white, rightrule=.15mm, coltitle=black, left=2mm, colframe=quarto-callout-important-color-frame, breakable, arc=.35mm, bottomtitle=1mm, toptitle=1mm, titlerule=0mm, leftrule=.75mm, colbacktitle=quarto-callout-important-color!10!white, title={Lemma 4.4 (Transformed PDE)}, toprule=.15mm, opacitybacktitle=0.6, bottomrule=.15mm, opacityback=0]

The function \(u(\tau, x)\) satisfies the diffusion equation:
\(\frac{\partial u}{\partial \tau} = \frac{1}{2}\sigma^2 \frac{\partial^2 u}{\partial x^2} + \left(r - \frac{\sigma^2}{2}\right) \frac{\partial u}{\partial x}\)

with initial condition: \(u(0, x) = (e^x - 1)^+\)

\end{tcolorbox}

\begin{tcolorbox}[enhanced jigsaw, colback=white, rightrule=.15mm, coltitle=black, left=2mm, colframe=quarto-callout-note-color-frame, breakable, arc=.35mm, bottomtitle=1mm, toptitle=1mm, titlerule=0mm, leftrule=.75mm, colbacktitle=quarto-callout-note-color!10!white, title=\textcolor{quarto-callout-note-color}{\faInfo}\hspace{0.5em}{Proof of Lemma 4.4}, toprule=.15mm, opacitybacktitle=0.6, bottomrule=.15mm, opacityback=0]

We need to compute the partial derivatives of \(V\) in terms of \(u\).
We have: \(V(t,S) = K e^{-r(T-t)} u(T-t, \ln(S/K))\)

Let \(\tau = T-t\) and \(x = \ln(S/K)\). Then:

\begin{align}
\frac{\partial V}{\partial t} &= K e^{-r\tau} \left[r u - \frac{\partial u}{\partial \tau}\right] \\
\frac{\partial V}{\partial S} &= K e^{-r\tau} \frac{1}{S} \frac{\partial u}{\partial x} \\
\frac{\partial^2 V}{\partial S^2} &= K e^{-r\tau} \frac{1}{S^2} \left[\frac{\partial^2 u}{\partial x^2} - \frac{\partial u}{\partial x}\right]
\end{align}

Substituting into the Black-Scholes PDE and simplifying (after dividing
by \(K e^{-r\tau}\)):
\(-\frac{\partial u}{\partial \tau} + \left(r - \frac{\sigma^2}{2}\right) \frac{\partial u}{\partial x} + \frac{1}{2}\sigma^2 \frac{\partial^2 u}{\partial x^2} = 0\)

Therefore:
\(\frac{\partial u}{\partial \tau} = \frac{1}{2}\sigma^2 \frac{\partial^2 u}{\partial x^2} + \left(r - \frac{\sigma^2}{2}\right) \frac{\partial u}{\partial x}\)

The initial condition at \(\tau = 0\) (i.e., \(t = T\)) is:
\(u(0, x) = \frac{(Ke^x - K)^+}{K} = (e^x - 1)^+\)

\end{tcolorbox}

\begin{tcolorbox}[enhanced jigsaw, colback=white, rightrule=.15mm, coltitle=black, left=2mm, colframe=quarto-callout-important-color-frame, breakable, arc=.35mm, bottomtitle=1mm, toptitle=1mm, titlerule=0mm, leftrule=.75mm, colbacktitle=quarto-callout-important-color!10!white, title={Theorem 4.3 (Solution via Fundamental Solution)}, toprule=.15mm, opacitybacktitle=0.6, bottomrule=.15mm, opacityback=0]

The solution to the transformed PDE is:
\(u(\tau, x) = \int_{-\infty}^{\infty} (e^y - 1)^+ G(\tau; x, y) dy\)

where \(G(\tau; x, y)\) is the fundamental solution:
\(G(\tau; x, y) = \frac{1}{\sqrt{2\pi\sigma^2\tau}} \exp\left(-\frac{(y - x - (r-\sigma^2/2)\tau)^2}{2\sigma^2\tau}\right)\)

\end{tcolorbox}

\hypertarget{step-8-computing-the-solution}{%
\subsection{Step 8: Computing the
Solution}\label{step-8-computing-the-solution}}

\begin{tcolorbox}[enhanced jigsaw, colback=white, rightrule=.15mm, coltitle=black, left=2mm, colframe=quarto-callout-important-color-frame, breakable, arc=.35mm, bottomtitle=1mm, toptitle=1mm, titlerule=0mm, leftrule=.75mm, colbacktitle=quarto-callout-important-color!10!white, title={Theorem 4.4 (Explicit Solution of the PDE)}, toprule=.15mm, opacitybacktitle=0.6, bottomrule=.15mm, opacityback=0]

After computing the integral (which follows the same steps as in the
martingale approach), we obtain:
\(u(\tau, x) = e^x \Phi(d_1) - \Phi(d_2)\)

where: \begin{align}
d_1 &= \frac{x + (r + \sigma^2/2)\tau}{\sigma\sqrt{\tau}} \\
d_2 &= \frac{x + (r - \sigma^2/2)\tau}{\sigma\sqrt{\tau}} = d_1 - \sigma\sqrt{\tau}
\end{align}

\end{tcolorbox}

\begin{tcolorbox}[enhanced jigsaw, colback=white, rightrule=.15mm, coltitle=black, left=2mm, colframe=quarto-callout-important-color-frame, breakable, arc=.35mm, bottomtitle=1mm, toptitle=1mm, titlerule=0mm, leftrule=.75mm, colbacktitle=quarto-callout-important-color!10!white, title={Theorem 4.5 (Black-Scholes Formula via PDE Method)}, toprule=.15mm, opacitybacktitle=0.6, bottomrule=.15mm, opacityback=0]

Transforming back to the original variables, the European call option
price is: \(V(t,S) = S \Phi(d_1) - K e^{-r(T-t)} \Phi(d_2)\)

where: \begin{align}
d_1 &= \frac{\ln(S/K) + (r + \sigma^2/2)(T-t)}{\sigma\sqrt{T-t}} \\
d_2 &= d_1 - \sigma\sqrt{T-t}
\end{align}

\end{tcolorbox}

\begin{tcolorbox}[enhanced jigsaw, colback=white, rightrule=.15mm, coltitle=black, left=2mm, colframe=quarto-callout-note-color-frame, breakable, arc=.35mm, bottomtitle=1mm, toptitle=1mm, titlerule=0mm, leftrule=.75mm, colbacktitle=quarto-callout-note-color!10!white, title=\textcolor{quarto-callout-note-color}{\faInfo}\hspace{0.5em}{Proof of Theorem 4.5}, toprule=.15mm, opacitybacktitle=0.6, bottomrule=.15mm, opacityback=0]

We have: \(V(t,S) = K e^{-r(T-t)} u(T-t, \ln(S/K))\)

The transformation from the \(u\) solution to the \(V\) solution follows
the same integration techniques as in the martingale approach. The key
insight is that both methods yield identical results, demonstrating the
equivalence of risk-neutral valuation and PDE approaches.

\end{tcolorbox}

\hypertarget{verification-and-conclusion}{%
\section{Verification and
Conclusion}\label{verification-and-conclusion}}

\begin{tcolorbox}[enhanced jigsaw, colback=white, rightrule=.15mm, coltitle=black, left=2mm, colframe=quarto-callout-important-color-frame, breakable, arc=.35mm, bottomtitle=1mm, toptitle=1mm, titlerule=0mm, leftrule=.75mm, colbacktitle=quarto-callout-important-color!10!white, title={Theorem 5.1 (Equivalence of Both Methods)}, toprule=.15mm, opacitybacktitle=0.6, bottomrule=.15mm, opacityback=0]

Both the martingale approach and the PDE approach yield the identical
Black-Scholes formula:
\(V(t,S) = S \Phi(d_1) - K e^{-r(T-t)} \Phi(d_2)\)

This demonstrates the deep connection between:

\begin{itemize}
\tightlist
\item
  Risk-neutral valuation (martingale theory)
\item
  Dynamic hedging and partial differential equations\\
\item
  Stochastic calculus and deterministic analysis
\end{itemize}

\end{tcolorbox}

\hypertarget{economic-interpretation}{%
\subsection{Economic Interpretation}\label{economic-interpretation}}

\begin{tcolorbox}[enhanced jigsaw, colback=white, rightrule=.15mm, coltitle=black, left=2mm, colframe=quarto-callout-tip-color-frame, breakable, arc=.35mm, bottomtitle=1mm, toptitle=1mm, titlerule=0mm, leftrule=.75mm, colbacktitle=quarto-callout-tip-color!10!white, title={Remark 5.1 (Economic Interpretation)}, toprule=.15mm, opacitybacktitle=0.6, bottomrule=.15mm, opacityback=0]

The Black-Scholes formula has a clear economic interpretation:

\begin{itemize}
\tightlist
\item
  \(S \Phi(d_1)\): Expected value of the stock if exercised, weighted by
  probability of exercise
\item
  \(K e^{-r(T-t)} \Phi(d_2)\): Present value of strike price, weighted
  by probability of exercise\\
\item
  \(\Phi(d_2)\): Risk-neutral probability that option finishes
  in-the-money
\item
  \(\Delta = \frac{\partial V}{\partial S} = \Phi(d_1)\): Hedge ratio
  (number of shares to hold per option)
\end{itemize}

\end{tcolorbox}

\hypertarget{key-insights}{%
\subsection{Key Insights}\label{key-insights}}

\begin{tcolorbox}[enhanced jigsaw, colback=white, rightrule=.15mm, coltitle=black, left=2mm, colframe=quarto-callout-tip-color-frame, breakable, arc=.35mm, bottomtitle=1mm, toptitle=1mm, titlerule=0mm, leftrule=.75mm, colbacktitle=quarto-callout-tip-color!10!white, title={Remark 5.2 (Key Insights)}, toprule=.15mm, opacitybacktitle=0.6, bottomrule=.15mm, opacityback=0]

The derivation reveals several fundamental insights:

\begin{enumerate}
\def\labelenumi{\arabic{enumi}.}
\tightlist
\item
  The option price depends only on the risk-free rate \(r\), not the
  actual expected return \(\mu\) of the stock
\item
  Perfect hedging is possible in continuous time with continuous trading
\item
  The same formula emerges from completely different mathematical
  approaches
\item
  The risk-neutral measure transforms the pricing problem into an
  expectation calculation
\end{enumerate}

\end{tcolorbox}

\hypertarget{greeks-and-risk-management}{%
\subsection{Greeks and Risk
Management}\label{greeks-and-risk-management}}

\begin{tcolorbox}[enhanced jigsaw, colback=white, rightrule=.15mm, coltitle=black, left=2mm, colframe=quarto-callout-note-color-frame, breakable, arc=.35mm, bottomtitle=1mm, toptitle=1mm, titlerule=0mm, leftrule=.75mm, colbacktitle=quarto-callout-note-color!10!white, title={Definition 5.1 (The Greeks)}, toprule=.15mm, opacitybacktitle=0.6, bottomrule=.15mm, opacityback=0]

The \textbf{Greeks} measure sensitivities of the option price to various
parameters:

\textbf{Delta}: \(\Delta = \frac{\partial V}{\partial S} = \Phi(d_1)\) -
Measures sensitivity to stock price changes - Represents the hedge ratio

\textbf{Gamma}:
\(\Gamma = \frac{\partial^2 V}{\partial S^2} = \frac{\phi(d_1)}{S\sigma\sqrt{T-t}}\)
- Measures rate of change of delta - Important for portfolio convexity

\textbf{Theta}:
\(\Theta = \frac{\partial V}{\partial t} = -\frac{S\phi(d_1)\sigma}{2\sqrt{T-t}} - rKe^{-r(T-t)}\Phi(d_2)\)
- Measures time decay - Always negative for long call options

\textbf{Vega}:
\(\nu = \frac{\partial V}{\partial \sigma} = S\phi(d_1)\sqrt{T-t}\) -
Measures sensitivity to volatility changes - Always positive for long
options

\textbf{Rho}:
\(\rho = \frac{\partial V}{\partial r} = K(T-t)e^{-r(T-t)}\Phi(d_2)\) -
Measures sensitivity to interest rate changes

where \(\phi(x) = \frac{1}{\sqrt{2\pi}}e^{-x^2/2}\) is the standard
normal density.

\end{tcolorbox}

\hypertarget{extensions-and-limitations}{%
\subsection{Extensions and
Limitations}\label{extensions-and-limitations}}

\begin{tcolorbox}[enhanced jigsaw, colback=white, rightrule=.15mm, coltitle=black, left=2mm, colframe=quarto-callout-warning-color-frame, breakable, arc=.35mm, bottomtitle=1mm, toptitle=1mm, titlerule=0mm, leftrule=.75mm, colbacktitle=quarto-callout-warning-color!10!white, title={Remark 5.3 (Model Assumptions and Limitations)}, toprule=.15mm, opacitybacktitle=0.6, bottomrule=.15mm, opacityback=0]

The Black-Scholes model relies on several strong assumptions:

\textbf{Assumptions:} - Constant risk-free rate and volatility - No
dividends - European exercise only - No transaction costs - Continuous
trading possible - Log-normal stock price distribution

\textbf{Limitations:} - Volatility smile/skew not captured - Jump risk
ignored - Early exercise features not handled - Transaction costs can be
significant - Liquidity constraints in practice

\textbf{Extensions:} - American options (early exercise) - Stochastic
volatility models (Heston, etc.) - Jump-diffusion models (Merton) -
Dividend-paying stocks - Time-dependent parameters

\end{tcolorbox}

\hypertarget{computational-implementation}{%
\subsection{Computational
Implementation}\label{computational-implementation}}

\begin{tcolorbox}[enhanced jigsaw, colback=white, rightrule=.15mm, coltitle=black, left=2mm, colframe=quarto-callout-note-color-frame, breakable, arc=.35mm, bottomtitle=1mm, toptitle=1mm, titlerule=0mm, leftrule=.75mm, colbacktitle=quarto-callout-note-color!10!white, title={Example 5.1 (Python Implementation)}, toprule=.15mm, opacitybacktitle=0.6, bottomrule=.15mm, opacityback=0]

\begin{Shaded}
\begin{Highlighting}[]
\ImportTok{import}\NormalTok{ numpy }\ImportTok{as}\NormalTok{ np}
\ImportTok{from}\NormalTok{ scipy.stats }\ImportTok{import}\NormalTok{ norm}

\KeywordTok{def}\NormalTok{ black\_scholes\_call(S, K, T, r, sigma):}
    \CommentTok{"""}
\CommentTok{    Calculate Black{-}Scholes call option price}
\CommentTok{    }
\CommentTok{    Parameters:}
\CommentTok{    S: Current stock price}
\CommentTok{    K: Strike price  }
\CommentTok{    T: Time to expiration}
\CommentTok{    r: Risk{-}free rate}
\CommentTok{    sigma: Volatility}
\CommentTok{    """}
\NormalTok{    d1 }\OperatorTok{=}\NormalTok{ (np.log(S}\OperatorTok{/}\NormalTok{K) }\OperatorTok{+}\NormalTok{ (r }\OperatorTok{+} \FloatTok{0.5}\OperatorTok{*}\NormalTok{sigma}\OperatorTok{**}\DecValTok{2}\NormalTok{)}\OperatorTok{*}\NormalTok{T) }\OperatorTok{/}\NormalTok{ (sigma}\OperatorTok{*}\NormalTok{np.sqrt(T))}
\NormalTok{    d2 }\OperatorTok{=}\NormalTok{ d1 }\OperatorTok{{-}}\NormalTok{ sigma}\OperatorTok{*}\NormalTok{np.sqrt(T)}
    
\NormalTok{    call\_price }\OperatorTok{=}\NormalTok{ S}\OperatorTok{*}\NormalTok{norm.cdf(d1) }\OperatorTok{{-}}\NormalTok{ K}\OperatorTok{*}\NormalTok{np.exp(}\OperatorTok{{-}}\NormalTok{r}\OperatorTok{*}\NormalTok{T)}\OperatorTok{*}\NormalTok{norm.cdf(d2)}
    
    \ControlFlowTok{return}\NormalTok{ call\_price}

\KeywordTok{def}\NormalTok{ calculate\_greeks(S, K, T, r, sigma):}
    \CommentTok{"""Calculate the Greeks"""}
\NormalTok{    d1 }\OperatorTok{=}\NormalTok{ (np.log(S}\OperatorTok{/}\NormalTok{K) }\OperatorTok{+}\NormalTok{ (r }\OperatorTok{+} \FloatTok{0.5}\OperatorTok{*}\NormalTok{sigma}\OperatorTok{**}\DecValTok{2}\NormalTok{)}\OperatorTok{*}\NormalTok{T) }\OperatorTok{/}\NormalTok{ (sigma}\OperatorTok{*}\NormalTok{np.sqrt(T))}
\NormalTok{    d2 }\OperatorTok{=}\NormalTok{ d1 }\OperatorTok{{-}}\NormalTok{ sigma}\OperatorTok{*}\NormalTok{np.sqrt(T)}
    
\NormalTok{    delta }\OperatorTok{=}\NormalTok{ norm.cdf(d1)}
\NormalTok{    gamma }\OperatorTok{=}\NormalTok{ norm.pdf(d1) }\OperatorTok{/}\NormalTok{ (S}\OperatorTok{*}\NormalTok{sigma}\OperatorTok{*}\NormalTok{np.sqrt(T))}
\NormalTok{    theta }\OperatorTok{=}\NormalTok{ (}\OperatorTok{{-}}\NormalTok{S}\OperatorTok{*}\NormalTok{norm.pdf(d1)}\OperatorTok{*}\NormalTok{sigma}\OperatorTok{/}\NormalTok{(}\DecValTok{2}\OperatorTok{*}\NormalTok{np.sqrt(T)) }
             \OperatorTok{{-}}\NormalTok{ r}\OperatorTok{*}\NormalTok{K}\OperatorTok{*}\NormalTok{np.exp(}\OperatorTok{{-}}\NormalTok{r}\OperatorTok{*}\NormalTok{T)}\OperatorTok{*}\NormalTok{norm.cdf(d2))}
\NormalTok{    vega }\OperatorTok{=}\NormalTok{ S}\OperatorTok{*}\NormalTok{norm.pdf(d1)}\OperatorTok{*}\NormalTok{np.sqrt(T)}
\NormalTok{    rho }\OperatorTok{=}\NormalTok{ K}\OperatorTok{*}\NormalTok{T}\OperatorTok{*}\NormalTok{np.exp(}\OperatorTok{{-}}\NormalTok{r}\OperatorTok{*}\NormalTok{T)}\OperatorTok{*}\NormalTok{norm.cdf(d2)}
    
    \ControlFlowTok{return}\NormalTok{ \{}\StringTok{\textquotesingle{}delta\textquotesingle{}}\NormalTok{: delta, }\StringTok{\textquotesingle{}gamma\textquotesingle{}}\NormalTok{: gamma, }\StringTok{\textquotesingle{}theta\textquotesingle{}}\NormalTok{: theta, }\StringTok{\textquotesingle{}vega\textquotesingle{}}\NormalTok{: vega, }\StringTok{\textquotesingle{}rho\textquotesingle{}}\NormalTok{: rho\}}

\CommentTok{\# Example usage}
\NormalTok{S0 }\OperatorTok{=} \DecValTok{100}    \CommentTok{\# Current stock price}
\NormalTok{K }\OperatorTok{=} \DecValTok{105}     \CommentTok{\# Strike price}
\NormalTok{T }\OperatorTok{=} \FloatTok{0.25}    \CommentTok{\# 3 months to expiration}
\NormalTok{r }\OperatorTok{=} \FloatTok{0.05}    \CommentTok{\# 5\% risk{-}free rate  }
\NormalTok{sigma }\OperatorTok{=} \FloatTok{0.2} \CommentTok{\# 20\% volatility}

\NormalTok{price }\OperatorTok{=}\NormalTok{ black\_scholes\_call(S0, K, T, r, sigma)}
\NormalTok{greeks }\OperatorTok{=}\NormalTok{ calculate\_greeks(S0, K, T, r, sigma)}

\BuiltInTok{print}\NormalTok{(}\SpecialStringTok{f"Call option price: $}\SpecialCharTok{\{}\NormalTok{price}\SpecialCharTok{:.2f\}}\SpecialStringTok{"}\NormalTok{)}
\BuiltInTok{print}\NormalTok{(}\SpecialStringTok{f"Delta: }\SpecialCharTok{\{}\NormalTok{greeks[}\StringTok{\textquotesingle{}delta\textquotesingle{}}\NormalTok{]}\SpecialCharTok{:.4f\}}\SpecialStringTok{"}\NormalTok{)}
\BuiltInTok{print}\NormalTok{(}\SpecialStringTok{f"Gamma: }\SpecialCharTok{\{}\NormalTok{greeks[}\StringTok{\textquotesingle{}gamma\textquotesingle{}}\NormalTok{]}\SpecialCharTok{:.4f\}}\SpecialStringTok{"}\NormalTok{)}
\end{Highlighting}
\end{Shaded}

\end{tcolorbox}

This complete treatment of the Black-Scholes model provides both
theoretical rigor and practical applicability, showing how mathematical
finance bridges pure mathematics and real-world financial applications.



\end{document}
